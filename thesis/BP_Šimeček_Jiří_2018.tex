% !TEX encoding = UTF-8 Unicode
% -*- coding: UTF-8; -*-
% vim: set fenc=utf-8

% arara: xelatex: {shell: true}
% arara: makeglossaries
% arara: biber
% arara: xelatex: {shell: true}
% arara: xelatex: {shell: true}

% options:
% thesis=B bachelor's thesis
% thesis=M master's thesis
% czech thesis in Czech language
% slovak thesis in Slovak language
% english thesis in English language
% hidelinks remove colour boxes around hyperlinks

\documentclass[thesis=B,czech]{./template/FITthesis}[2012/06/26]

\usepackage[utf8]{inputenc} % LaTeX source encoded as UTF-8
\usepackage{polyglossia} % multilanguage support
\usepackage{csquotes} % czech quotes
\usepackage{xevlna} % automatické nezalomitelné mezery
\usepackage{metalogo} % xelatex logo nad more
\usepackage{minted} % zvýraznění kódu pomocí python Pygments
\usepackage{graphicx} % graphics files inclusion
% \usepackage{amsmath} % advanced maths
% \usepackage{amssymb} % additional math symbols
\usepackage{dirtree} % directory tree visualisation
\usepackage{enumitem} % prefix for enumerate items
\usepackage{multirow} % table rowspan

% list of acronyms
\usepackage[nomain,acronym,nonumberlist,toc,numberedsection=autolabel]{glossaries}
\iflanguage{czech}{\renewcommand*{\acronymname}{Seznam použitých zkratek}}{}

\makeglossaries

\usepackage[style=iso-numeric]{biblatex}
\addbibresource{partials/analyza/technologie.bib}
\addbibresource{partials/analyza/algoritmy.bib}
\addbibresource{partials/analyza/pushpull.bib}
\addbibresource{partials/analyza/existujiciReseni.bib}

\addbibresource{partials/navrh/zdroje.bib}

\newcommand{\tg}{\mathop{\mathrm{tg}}} %cesky tangens
\newcommand{\cotg}{\mathop{\mathrm{cotg}}} %cesky cotangens

\setmainlanguage{czech}

\renewcommand{\listingscaption}{Výpis kódu}
\renewcommand{\listoflistingscaption}{Seznam výpisů kódu}

% % % % % % % % % % % % % % % % % % % % % % % % % % % % % % 
% ODTUD DAL VSE ZMENTE
% % % % % % % % % % % % % % % % % % % % % % % % % % % % % %

\department{Katedra softwarového inženýrství}
\title{Webový nástroj pro kolaborativní editaci textů}
\authorGN{Jiří} %(křestní) jméno (jména) autora
\authorFN{Šimeček} %příjmení autora
\authorWithDegrees{Jiří Šimeček} %jméno autora včetně současných akademických titulů
\author{Jiří Šimeček} %jméno autora bez akademických titulů
\supervisor{Ing. Petr Špaček, Ph.D.}
% TODO: poděkování
\acknowledgements{Chtěl bych poděkovat\ldots}
% TODO: abstrakt
\abstractCS{Tato práce se zabývá problémem kolaborativní editace textů a porovnává jednotlivé známé algoritmy, které tento problém řeší. Dále se zabývá návrhem a implementací prototypu pomocí jednoho z vybraných algortmů.}
\abstractEN{Sem doplňte ekvivalent abstraktu Vaší práce v~angličtině.}
\placeForDeclarationOfAuthenticity{V~Praze}
\declarationOfAuthenticityOption{4} %volba Prohlášení (číslo 1-6)
% TODO: klíčová slova
\keywordsCS{návrh webové aplikace, kolaborativní editace textů, web v~reálném čase, Javascript, ReactJS, NodeJS}
\keywordsEN{Nahraďte seznamem klíčových slov v angličtině oddělených čárkou.}
% \website{https://editor.simecekjiri.cz/} %volitelná URL práce, objeví se v tiráži - úplně odstraňte, nemáte-li URL práce

% usage: \gls{FIT}

\newacronym{HTTP}{HTTP}{Hypertext Transfer Protocol}
\newacronym{API}{API}{Application Programming Interface}
\newacronym{OT}{OT}{Operační transformace}
\newacronym{DS}{DS}{Diferenciální synchronizace}
\newacronym{HTML}{HTML}{Hypertext Markup Language}
\newacronym{W3C}{W3C}{The World Wide Web Consortium}
\newacronym{IETF}{IETF}{Internet Engineering Task Force}
\newacronym{WHATWG}{WHATWG}{Web Hypertext Application Technology Working Group}
\newacronym{WWW}{WWW}{World Wide Web}
\newacronym{ECMA}{ECMA}{European Computer Manufacturer's Association}
\newacronym{I/O}{I/O}{vstupní/výstupní}
\newacronym{SQL}{SQL}{Structured Query Language}
\newacronym{NoSQL}{NoSQL}{Not only \gls{SQL}}
\newacronym{SŘBD}{SŘBD}{Systém Řízení Báze Dat}
\newacronym{DBMS}{DBMS}{Database Management System}
\newacronym{BSON}{BSON}{Binary \gls{JSON}}
\newacronym{JSON}{JSON}{JavaScript Object Notation}
\newacronym{ORM}{ORM}{Object-relational mapping}
\newacronym{ODM}{ODM}{Object-document mapping}
\newacronym{SVN}{SVN}{Apache Subversion}
\newacronym{WebRTC}{WebRTC}{Web Real-Time Communication}
\newacronym{WYSIWYG}{WYSIWYG}{\enquote{co vidíš, to dostaneš}}
\newacronym{CSS3}{CSS3}{Cascading Style Sheets verze 3}
\newacronym{TCP/IP}{TCP/IP}{Transmission Control Protocol/Internet Protocol}
\newacronym{DB}{DB}{Databáze}
\newacronym{REST}{REST}{Representational state transfer}
\newacronym{MVC}{MVC}{Model–view–controller}
\newacronym{URL}{URL}{Uniform Resource Locator}


\begin{document}

    % !TEX encoding = UTF-8 Unicode
% -*- coding: UTF-8; -*-
% vim: set fenc=utf-8

% aktuálnost
Webové aplikace, které komunikují s uživatelem v reálném čase dnes nabývají na oblibě.
Uživatel již běžně očekává, že se mu na webových stránkách zobrazují nejrůznější upozornění, či se dokonce aktualizují celé části webové stránky.
Nově se začínají objevovat webové nástroje pro kolaborativní spolupráci nad texty (případně nad jinými multimédii), které kombinují myšlenku tvorby obsahu webu uživateli a právě odezvu aplikace v reálném čase.

% významnost
Výstupem této práce je všeobecně nasaditelná komponenta, která bude použita jako jedna z komponent projektu webového IDE s pracovním názvem Laplace-IDE.
Komponenta je také určena pro potřeby vývojářů, kteří chtějí vytvořit kolaborativní webový nástroj a nechtějí ho vytvářet od nuly.

% motivace
Toto téma jsem si zvolil, jelikož většina doposud existujících kolaborativních textových nástojů je postavena nad uzavřeným kódem nebo nad knihovnami, jejichž vývoj byl ukončen.
Neexistují tak nástroje, či knihovny, které by bylo možné bez větších problémů použít pro vlastní projekty.

% zaměření
V této práci se zabývám analýzou problému kolaborativní spolupráce, porovnáním a výběrem vhodných existujících algoritmů a technologií, návrhem znovupoužitelné komponenty a implementací prototypu včetně navržené komponenty.

% struktura práce (záleží na jednotlivých kapitolách)
Tato práce dále pokračuje v následující struktuře:
Nejprve se v části 1 zabývám analýzou a výběrem vhodných algoritmů, z které pak přecházím k návrhu komponenty a prototypu v části 2.
Navržený prototyp dále v části 3 implementuji a na konec nad výslednou implementací v bodě 4 provádím uživatelské testování.


    % !TEX encoding = UTF-8 Unicode
% -*- coding: UTF-8; -*-
% vim: set fenc=utf-8
%\inputencoding{utf8}

\chapter{Cíl práce}

% Cílem rešeršní části práce je
Cílem rešeršní části práce je analýza požadavků, následné seznámení se zadanými technologiemi a rozbor existujících webových komunikačních protokolů.
Dalším cílem je analýza problematiky kolaborativní editace textů a nejčastěji používaných synchronizačních algoritmů.
Ale také rozbor existujících aplikací, které umožňují editaci ve skutečném čase, a analýza doménového modelu.


% Dokumentace a best practise zadaných technologií (HTML5, Javascript, ReactJS, NodeJS)
% Analýza a výběr algoritmů pro kolaborativní spolupráci (OT vs DS)
% Analýza a výběr existujících webových real-time protokolů (http server push - WebSocket, long pooling, pushlet)


% Cílem praktické části je
Cílem praktické části je navržení modelu pro uložení dat a prototypu komponenty kolaborativního textového editoru ve skutečném čase.
Další cílem je implementace a použití navržené komponenty.
A následně uživatelské otestování a vyhodnocení kvality implementovaného řešení.

% Navrhněte model uložení textů, model pro uložení informací o uživatelích a model pro editační změnu.
% Na základě návrhu implementujte prototyp takového nástroje.
% Proveďte uživatelské otestování výsledku a vyhodnoťte kvality a nedostatky vašeho řešení.


    %% !TEX encoding = UTF-8 Unicode
% -*- coding: UTF-8; -*-
% vim: set fenc=utf-8
%\inputencoding{utf8}

\chapter{Testování}\label{ch:testování}

Zatím se jedná jen o popsanou struktury kapitoly.

\subsection{Uživatelské testování}

Průchod 3 uživatelů systémem (scénaře podle uživatelských případů), postřehy z testování a shrnutí (UX není přímým cílem práce, jde především o analýzu problému a návrh jeho řešení).

\subsection{Testování rychlosti}

Rychlost synchronizace mezi více uživateli (porovnání rychlosti s přibývajícím počtem uživatelů).

Test prototypu při velkém počtu uživatel (cca 100 připojených uživatel na jeden dokument).
Postřehy a shrnutí (možnosti optimalizace a co to znamená pro další škálování).


    %% !TEX encoding = UTF-8 Unicode
% -*- coding: UTF-8; -*-
% vim: set fenc=utf-8
%\inputencoding{utf8}

\chapter{Testování}\label{ch:testování}

Zatím se jedná jen o popsanou struktury kapitoly.

\subsection{Uživatelské testování}

Průchod 3 uživatelů systémem (scénaře podle uživatelských případů), postřehy z testování a shrnutí (UX není přímým cílem práce, jde především o analýzu problému a návrh jeho řešení).

\subsection{Testování rychlosti}

Rychlost synchronizace mezi více uživateli (porovnání rychlosti s přibývajícím počtem uživatelů).

Test prototypu při velkém počtu uživatel (cca 100 připojených uživatel na jeden dokument).
Postřehy a shrnutí (možnosti optimalizace a co to znamená pro další škálování).


    % !TEX encoding = UTF-8 Unicode
% -*- coding: UTF-8; -*-
% vim: set fenc=utf-8
%\inputencoding{utf8}

\chapter{Testování}\label{ch:testování}

Zatím se jedná jen o popsanou struktury kapitoly.

\subsection{Uživatelské testování}

Průchod 3 uživatelů systémem (scénaře podle uživatelských případů), postřehy z testování a shrnutí (UX není přímým cílem práce, jde především o analýzu problému a návrh jeho řešení).

\subsection{Testování rychlosti}

Rychlost synchronizace mezi více uživateli (porovnání rychlosti s přibývajícím počtem uživatelů).

Test prototypu při velkém počtu uživatel (cca 100 připojených uživatel na jeden dokument).
Postřehy a shrnutí (možnosti optimalizace a co to znamená pro další škálování).


    \input{partials/testing}

    % !TEX encoding = UTF-8 Unicode
% -*- coding: UTF-8; -*-
% vim: set fenc=utf-8
%\inputencoding{utf8}

\begin{conclusion}

    V rešeršní části práce byly nejprve analyzovány funkční a nefunkční požadavky tohoto nástroje a byl sestaven jeho doménový model.
    Dále byly stručně představeny zadané technologie a existující webové komunikační protokoly včetně jejich využití a nevýhod.
    Také byla představena problematika synchronizace textů ve skutečném čase spolu s nejčastěji používanými synchronizačními algoritmy, operační transformace a diferenciální synchronizace.
    A rešeršní část byla zakončena představením existujících nástrojů pro kolaborativní editaci textů.

    V praktické části práce byl navržen model pro uložení dat a komponenta kolaborativního textového editoru využívající algoritmu operační transformace.
    Dále byl navržen a implementován prototyp aplikace implementující navrženou komponentu kolaborativního textového editoru.
    Uživatelské rozhraní tohoto prototypu bylo úspěšně uživatelsky otestováno a implementace komponenty editoru v rámci prototypu byla otestována výkonnostně.

    Vzniklý prototyp aplikace umožňuje jednoduchou kolaborativní editaci textů, obsahuje jednoduchou správu dokumentů a uživatelských účtů.
    Jednotlivé dokumenty je možné sdílet pomocí jejich veřejných odkazů a to i pouze v režimu náhledu, kdy je dokument také synchronizován ve skutečném čase.

    V budoucnosti by bylo možné navrženou komponentu editoru zbavit závislostí na knihovnách třetích stran (OT.js a Material-UI) a publikovat jako samostatný balíček do npm (JavaScript package manager) registru.
    Takto implementovanou knihovnu by mohli jednoduše využívat i ostatní vývojáři, kteří chtějí vytvořit kolaborativní nástroj s podobnými vlastnostmi jako má implementovaný prototyp aplikace.

    % uvedení cílů či zaměření práce
    % způsob a míra splnění cílů
    % optional: výhled do budoucna
\end{conclusion}


    \printbibliography

    \appendix

    \printglossaries

    \chapter{Obsah přiloženého CD}\label{ch:obsahPrilozenehoCd}

    % TODO: upravte podle skutecnosti

    \begin{figure}
        \dirtree{%
        .1 readme.txt\DTcomment{stručný popis obsahu CD}.
        .1 exe\DTcomment{adresář se spustitelnou formou implementace}.
        .1 src.
        .2 impl\DTcomment{zdrojové kódy implementace}.
        .2 thesis\DTcomment{zdrojová forma práce ve formátu \LaTeX{}}.
        .1 text\DTcomment{text práce}.
        .2 thesis.pdf\DTcomment{text práce ve formátu PDF}.
        .2 thesis.ps\DTcomment{text práce ve formátu PS}.
        }
    \end{figure}

\end{document}
