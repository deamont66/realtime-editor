% !TEX encoding = UTF-8 Unicode
% -*- coding: UTF-8; -*-
% vim: set fenc=utf-8
%\inputencoding{utf8}

\chapter{Realizace}\label{ch:realizace}

Jednotlivé zajímavé části prototypu (zatím se jedná jen o popsanou struktury kapitoly).

\section{Zveřejnění klientské části}\label{sec:zveřejněníKlientskéČásti}

\subsection{Sdílení veřejných odkazů na dokumenty}

Server rendering meta tagů, knihovna react-helmet pro jejich aktualizaci.

\section{Autentizace}

Využití knihovny passport.js.
Autentizace pomocí ČVUT hesla a sociálních sítí (OAuth2).

\subsection{Zapomatování uživatele}

Cookie remember me token a jeho konzumace/generace při každém požadavku (ochrana cookie hijacking).

\subsection{Přihlášení pomocí hesla}

\subsubsection{Uložení hesla}

Použití hashovací funkce bcrypt a jeho výhody oproti použití \enquote{rychlých} hashovacích funkcí jako jsou funkce standartu sha (dedikovaný hardware a špatně nastavitelná obtížnost výpočtu).

\subsubsection{Odhad obtížnosti hesla}

Knihovna zxcvb od společnost Dropbox.
Problém s vynucování speciálních znaků v heslech (například hesla typu P@ssw0rd, které nemají téměř žádnou přidanou obtížnost díky použití známých substitucí).

\section{Autorizace}

Třída/metody \texttt{DocumentVoter} a řešení práv pro přístup k dokumentům.

Problém při zobrazení editoru (klient neví jestli dokument vůbec existuje nebo zda-li k němu má uživatel oprávnění) a jeho částí.

\section{Překlady}

Knihovna i18next a rect-i18next, asynchronní načítání překladů.

\section{Materail desing}

Knihovna material-ui ve verzi 1+, material design doporučení přímo od Google.
Co to řeší (konzistence a přístupnost UX), jak jsem to použil.

