% !TEX encoding = UTF-8 Unicode
% -*- coding: UTF-8; -*-
% vim: set fenc=utf-8
\inputencoding{utf8}

% aktuálnost
Webové aplikace, které komunikují s uživatelem v reálném čase, jsou dnes ve velké oblibě.
Uživatel například běžně očekává, že se mu na webové stránce v reálném čase zobrazují upozornění o nové příchozí zprávě, či se zpráva objeví sama o sobě.
Nově se začínají objevovat webové nástroje pro kolaborativní spolupráci, které kombinují myšlenku tvroby obsahu webu uživatelem a právě odezvu aplikace v reálném čase.
V této práci jsem se zaměřil právě na jeden takový nástroj, kterým je webová aplikace pro kolaborativní editaci textů.

% TODO: významnost

% Práce bude určena pro
% Laplace-IDE, pro koho to vůbec bude? (potřeba zjistit \ldots)

% motivace
% Toto téma jsem si vybral, protože\ldots (významnost + použití na cvičení programování)
% Toto téma jsem si zvolil, jelikož jsem nenalezl doposud dostatečný nástroj, který by poskytoval všechny mnou hledané možnosti nastavení a zároveň fungoval.

% zaměření



% struktura práce
