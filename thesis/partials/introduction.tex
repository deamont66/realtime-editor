% !TEX encoding = UTF-8 Unicode
% -*- coding: UTF-8; -*-
% vim: set fenc=utf-8

% aktuálnost
Webové aplikace, které komunikují s uživatelem v reálném čase dnes nabývají na oblibě.
Uživatel již běžně očekává, že se mu na webových stránkách zobrazují nejrůznější upozornění, či se dokonce aktualizují celé části webové stránky.
Nově se začínají objevovat webové nástroje pro kolaborativní spolupráci nad texty (případně nad jinými multimédii), které kombinují myšlenku tvorby obsahu webu uživateli a právě odezvu aplikace v reálném čase.

% významnost
Výstupem této práce je všeobecně nasaditelná komponenta, která bude použita jako jedna z komponent projektu webového IDE s pracovním názvem Laplace-IDE.
Komponenta je také určena pro potřeby vývojářů, kteří chtějí vytvořit kolaborativní webový nástroj a nechtějí ho vytvářet od nuly.

% motivace
Toto téma jsem si zvolil, jelikož většina doposud existujících kolaborativních textových nástojů je postavena nad uzavřeným kódem nebo nad knihovnami, jejichž vývoj byl ukončen.
Neexistují tak nástroje, či knihovny, které by bylo možné bez větších problémů použít pro vlastní projekty.

% zaměření
V této práci se zabývám analýzou problému kolaborativní spolupráce, porovnáním a výběrem vhodných existujících algoritmů a technologií, návrhem znovupoužitelné komponenty a implementací prototypu včetně navržené komponenty.

% struktura práce (záleží na jednotlivých kapitolách)
Tato práce dále pokračuje v následující struktuře:
Nejprve se v části 1 zabývám analýzou a výběrem vhodných algoritmů, z které pak přecházím k návrhu komponenty a prototypu v části 2.
Navržený prototyp dále v části 3 implementuji a na konec nad výslednou implementací v bodě 4 provádím uživatelské testování.
