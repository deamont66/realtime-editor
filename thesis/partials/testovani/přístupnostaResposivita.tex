% !TEX encoding = UTF-8 Unicode
% -*- coding: UTF-8; -*-
% vim: set fenc=utf-8
%\inputencoding{utf8}

\section{Webová použitelnost a podpora}\label{sec:webováPoužitelnostAPodpora}

Jak jsem již zmiňoval v sekci~\ref{sec:uživatelskéRozhraní} o uživatelském rozhraní, aplikace se řídí zásadami návrhu uživatelského rozhraní Material Design.
Tyto zásady hovoří i mimo jiné o použitelnost a přístupnosti webových aplikací.

Implementovaný prototyp aplikace byl otestován pomocí nástroje Google Lighthouse, nástroje s veřejným zdrojovým kódem, který slouží k testování přístupnosti webových aplikací.
Tento nástroj aplikace hodnotí v různých kategoriích bodovým ohodnocením mezi 0 až 100 body.
Implementovaný prototyp aplikace v tomto nástroji dosáhl hodnocení 100 bodů z přístupnosti, 94 bodů z doporučených postupů (anglicky Best practises) a 89 bodů z kategorie \gls{SEO}.
Stržené body z kategorie doporučených postupů jsou způsobeny nedostatečnou podporou \gls{HTTP}/2 prostředím, kde je prototyp aplikace nasazen.
Stržené body z kategorie \gls{SEO} jsou způsobeny dynamickým výpočtem velikosti písma uvnitř použité knihovny \texttt{Material-UI}.

Prototyp aplikace podporuje poslední stabilní verze hlavních platforem a webových prohlížečů.
Podpora verzí jednotlivých webových prohlížečů je dána jejich podporou použité knihovny \texttt{Material-UI} (viz sekce~\ref{subsec:materialDesign}).

Prototyp aplikace byl otestován a je funkční i několik verzí zpět, kde je limitujícím faktorem především podpora \acrshort{CSS} vlastnosti \texttt{Flex}.
Tato vlastnost je používána pro rozvržení rozhraní webové aplikace a bez její podpory se obsah nebude zobrazovat správně.
Podporované verze webových prohlížečů, úspěšně otestované verze a oficiální verze podpory vlastnosti \texttt{Flex} jsou zobrazené v tabulce~\ref{tab:verzeProhlížečů}.

\begin{table}[ht!]
    \centering
    \caption{Podporované verze webových prohlížečů}
    \label{tab:verzeProhlížečů}
    \begin{tabular}{l|ccccc}
        & IE & Edge & Firefox & Chrome & Safari \\ \hline
        Podpora prototypu aplikace & $11$ & $\geq 14$ & $\geq 45$ & $\geq 49$ & $\geq 10$ \\
        Úspěšně testované verze & $11$ & $\geq 14$ & $\geq 28$ & $\geq 39$ & $\geq 10$ \\
        Oficiální podpora vlastnosti Flex & $11$ & $\geq 12$ & $\geq 20$ & $\geq 29$ & $\geq 9$
    \end{tabular}
\end{table}
