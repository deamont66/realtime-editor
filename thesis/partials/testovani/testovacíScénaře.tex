% !TEX encoding = UTF-8 Unicode
% -*- coding: UTF-8; -*-
% vim: set fenc=utf-8
%\inputencoding{utf8}

\subsection{Testovací scénáře}\label{subsec:testovacíScénáře}

Testování může obsahovat více testovacích scénářů.
Scénáře obsahují jejich účel, čas, podmínky, kroky a očekávaný výsledek.
Účastník testu nesmí mít osobní zájem na jeho výsledku, musí být objektivní a nebát se cokoliv říct.
Přísedící musí na případné otázky účastníka odpovídat tak, aby nenarušil výsledky testu.

Účastník je seznámen s účelem scénáře a dále postupuje podle jeho kroků.
Během průchodu každého scénáře popisuje svou interakci s aplikací a své myšlenky, přísedící pořizuje poznámky či průběh testování nahrává.
Po skončení průchodu jsou poznámky zpracovány a výsledek je porovnán s očekávaným výsledkem scénáře.

Jednotlivé testovací scénáře reflektují stejnojmenné uživatelské případy ze sekce~\ref{sec:uzivatelskePripady}.

\paragraph{TC1 -- Vytvoření uživatelského účtu}

Účelem toho to testovacího scénáře je vytvoření nového uživatelského účtu.
Před začátkem průchodu je aplikace ve stavu, kdy není přihlášen žádný uživatel a je zobrazen přihlašovací formulář.
Doba průchodu celého scénáře by neměla být delší než 5 minut.

Kroky scénáře:
\begin{enumerate}
    \item Účastník klikne na odkaz \texttt{Registrovat se}.
    \item Vyplní registrační formulář.
    \item Formulář odešle kliknutím na tlačítko \texttt{Registrovat se}.
\end{enumerate}

Očekávaný výsledek po průchodu scénáře:
\begin{enumerate}
    \item Je vytvořen nový uživatelský účet se zadanými údaji.
    \item Na zadaný email je odeslán uvítací email.
    \item Vytvořený uživatel je přihlášen do aplikace.
    \item Je zobrazen prázdný seznam vytvořených dokumentů spolu s tlačítkem pro vytvoření nového dokumentu.
\end{enumerate}

\paragraph{TC2 -- Přihlášení do aplikace}

Účelem toho to testovacího scénáře je přihlášení uživatele pomocí již vytvořeného uživatelského účtu.
Před začátkem průchodu je aplikace ve stavu, kdy není přihlášen žádný uživatel a je zobrazen přihlašovací formulář.
Doba průchodu celého scénáře by neměla být delší než 5 minut.

Kroky scénáře:
\begin{enumerate}
    \item Účastník vyplní přihlašovací formulář.
    \item Formulář odešle kliknutím na tlačítko \texttt{Přihlásit se}.
\end{enumerate}

Očekávaný výsledek po průchodu scénáře:
\begin{enumerate}
    \item Uživatel je přihlášen do aplikace.
    \item Je zobrazen seznam vytvořených dokumentů spolu s tlačítkem pro vytvoření nového dokumentu.
\end{enumerate}

\paragraph{TC3 -- Změna uživatelských údajů}

Účelem toho to testovacího scénáře je změna uživatelských údajů přihlášeného uživatele.
Před začátkem průchodu je aplikace ve stavu, kdy je uživatel přihlášen a je zobrazen seznam vytvořených dokumentů.
Doba průchodu celého scénáře by neměla být delší než 5 minut.

Kroky scénáře:
\begin{enumerate}
    \item Účastník otevře vyskakovací menu s možnosti kliknutím na ikonu uživatele.
    \item Pokračuje kliknutím na tlačítko \texttt{Nastavení}.
    \item Vyplní formulář pro změnu uživatelských údajů.
    \item Formulář odešle kliknutím na tlačítko \texttt{Uložit změny}.
\end{enumerate}

Očekávaný výsledek po průchodu scénáře:
\begin{enumerate}
    \item Uživatelovi údaje jsou změněny.
    \item Je zobrazeno plovoucí upozornění o výsledku akce.
\end{enumerate}

\paragraph{TC4 -- Změna výchozího nastavení dokumentů}

Účelem toho to testovacího scénáře je změna výchozího nastavení pro nově vytvořené dokumenty.
Před začátkem průchodu je aplikace ve stavu, kdy je uživatel přihlášen a je zobrazen seznam vytvořených dokumentů.
Doba průchodu celého scénáře by neměla být delší než 10 minut.

Kroky scénáře:
\begin{enumerate}
    \item Účastník otevře vyskakovací menu s možnosti kliknutím na ikonu uživatele.
    \item Pokračuje kliknutím na tlačítko \texttt{Nastavení}.
    \item Přepne se do požadovaného nastavení kliknutím na záložku \texttt{Nastavení dokumentů}.
    \item Vyplní formulář s výchozím nastavením pro nově vytvořené dokumenty.
    \item Formulář odešle kliknutím na tlačítko \texttt{Uložit změny}.
\end{enumerate}

Očekávaný výsledek po průchodu scénáře:
\begin{enumerate}
    \item Výchozí nastavení údaje pro nově vytvořené dokumenty je uloženo.
    \item Je zobrazeno plovoucí upozornění o výsledku akce.
\end{enumerate}


\paragraph{TC5 -- Zobrazení seznamu dokumentů}

Účelem toho to testovacího scénáře je zobrazení seznamů vytvořených, sdílených a posledních dokumentů.
Před začátkem průchodu je aplikace ve stavu, kdy je uživatel přihlášen a je zobrazen seznam vytvořených dokumentů.
Doba průchodu celého scénáře by neměla být delší než 5 minut.

Kroky scénáře:
\begin{enumerate}
    \item Účastník otevře navigační menu kliknutím na ikonu navigace.
    \item Pokračuje kliknutím na tlačítka \texttt{Mé dokumenty}, \texttt{Sdílené} a \texttt{Poslední}.
\end{enumerate}

Očekávaný výsledek po průchodu scénáře:
\begin{enumerate}
    \item Postupně jsou zobrazeny všechny tři seznamy dokumentů.
\end{enumerate}


\paragraph{TC6 -- Vytvoření nového dokumentu}

Účelem toho to testovacího scénáře je vytvoření nového dokumentu a nastavení jeho jména.
Před začátkem průchodu je aplikace ve stavu, kdy je uživatel přihlášen a je zobrazen seznam vytvořených dokumentů.
Doba průchodu celého scénáře by neměla být delší než 5 minut.

Kroky scénáře:
\begin{enumerate}
    \item Účastník klikne na tlačítko \texttt{Vytvořit nový dokument}.
    \item Vyplní jméno dokumentu.
    \item Jméno uloží kliknutím na tlačítko s ikonou uložit.
\end{enumerate}

Očekávaný výsledek po průchodu scénáře:
\begin{enumerate}
    \item Nový dokument je vytvořen.
    \item Je zobrazen textový editor nově vytvořeného dokumentu.
    \item Dokument má nastavené vyplněné jméno.
\end{enumerate}

\paragraph{TC7 -- Otevření textového editoru}

Účelem toho to testovacího scénáře je zobrazení textového editoru pro již existujícího sdíleného dokumentu.
Před začátkem průchodu je aplikace ve stavu, kdy je uživatel přihlášen a je zobrazen seznam vytvořených dokumentů.
Doba průchodu celého scénáře by neměla být delší než 5 minut.

Kroky scénáře:
\begin{enumerate}
    \item Účastník otevře navigační menu kliknutím na ikonu navigace.
    \item Zobrazí seznam sdílených dokumentů kliknutím na tlačítko \texttt{Sdílené}.
    \item Pokračuje kliknutím na název vybraného dokumentu v seznamu.
\end{enumerate}

Očekávaný výsledek po průchodu scénáře:
\begin{enumerate}
    \item Je zobrazen textový editor vybraného dokumentu.
\end{enumerate}

\paragraph{TC8 -- Změna nastavení dokumentu}

Účelem toho to testovacího scénáře je změna nastavení již vytvořeného dokumentu.
Před začátkem průchodu je aplikace ve stavu, kdy je uživatel přihlášen a je zobrazen seznam vytvořených dokumentů.
Doba průchodu celého scénáře by neměla být delší než 10 minut.

Kroky scénáře:
\begin{enumerate}
    \item Účastník klikne na název vybraného dokumentu v seznamu.
    \item Pokračuje kliknutím na tlačítko nastavení dokumentu s ikonou ozubeného kola.
    \item Upraví nastavení dokumentu.
\end{enumerate}

Očekávaný výsledek po průchodu scénáře:
\begin{enumerate}
    \item Je zobrazen textový editor vybraného dokumentu a otevřeno postranní menu s jeho nastavením.
    \item Změny nastavení byly ve skutečném čase uloženy a aplikovány na otevřený editor.
\end{enumerate}

\paragraph{TC9 -- Přizvání uživatele k dokumentu}

Účelem toho to testovacího scénáře je přizvání dalšího uživatel k editaci již vytvořeného dokumentu.
Před začátkem průchodu je aplikace ve stavu, kdy je uživatel přihlášen a je zobrazen seznam vytvořených dokumentů.
Doba průchodu celého scénáře by neměla být delší než 10 minut.

Kroky scénáře:
\begin{enumerate}
    \item Účastník klikne na název vybraného dokumentu v seznamu.
    \item Pokračuje kliknutím na tlačítko \texttt{Sdílet}.
    \item Zobrazí formulář pro vytvoření nové pozvánky kliknutím na tlačítko \texttt{Nová pozvánka}.
    \item Vyplní formulář pro vytvoření nové pozvánky.
    \item Formulář odešle kliknutím na tlačítko \texttt{Pozvat}.
\end{enumerate}

Očekávaný výsledek po průchodu scénáře:
\begin{enumerate}
    \item Je zobrazen textový editor vybraného dokumentu a otevřeno postranní menu s možnostmi jeho sdílení.
    \item Pozvaný uživatel vidí dokument ve svém seznamu sdílených dokumentů.
    \item Pozvaný uživatel se zobrazil v seznamu pozvaných uživatel v otevřeném postranním menu.
\end{enumerate}

\paragraph{TC10 -- Komunikační vlákno dokumentu}

Účelem toho to testovacího scénáře je zobrazení komunikačního vlákna již vytvořeného dokumentu a přidání nové zprávy.
Před začátkem průchodu je aplikace ve stavu, kdy je uživatel přihlášen a je zobrazen seznam vytvořených dokumentů.
Doba průchodu celého scénáře by neměla být delší než 5 minut.

Kroky scénáře:
\begin{enumerate}
    \item Účastník klikne na název vybraného dokumentu v seznamu.
    \item Pokračuje kliknutím na tlačítko komunikačního vlákna s ikonou konzervační bubliny.
    \item Vyplní pole pro text nové zprávy.
    \item Zprávu odešle kliknutím na tlačítko s ikonou odeslat.
\end{enumerate}

Očekávaný výsledek po průchodu scénáře:
\begin{enumerate}
    \item Je zobrazen textový editor vybraného dokumentu a otevřeno postranní menu s komunikačním vláknem dokumentu.
    \item Odeslaná zpráva se zobrazila v komunikačním vláknem dokumentu.
\end{enumerate}
