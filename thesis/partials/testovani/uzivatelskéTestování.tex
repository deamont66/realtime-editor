% !TEX encoding = UTF-8 Unicode
% -*- coding: UTF-8; -*-
% vim: set fenc=utf-8
%\inputencoding{utf8}

\section{Uživatelské testování}\label{sec:uživatelskéTestování}

Uživatelské testování je nejčastější metodou testování přístupnosti aplikací.
Cílem je odkrýt závažné problémy aplikace, které mohou uživatelům působit problémy s používáním aplikace.
Tetování spočívá v pozorování samotných uživatelů a jejich interakce s aplikací, jak se snaží dosáhnout nějakého předem určeného cíle.

Uživatelské testování je zahájeno vysvětlením účelu jednotlivých scénářů, následováno testováním samotným, zpracováním výsledků a je zpravidla zakončeno uvedením výsledků.
Uvedení výsledků je důležité pro testované uživatele, protože v nich vzbuzuje pocit, že vývojářům aplikace skutečně pomohli.~\cite{book:userTesting}

% !TEX encoding = UTF-8 Unicode
% -*- coding: UTF-8; -*-
% vim: set fenc=utf-8
%\inputencoding{utf8}

\subsection{Testovací scénáře}\label{subsec:testovacíScénáře}

Testování může obsahovat více testovacích scénářů.
Scénáře obsahují jejich účel, čas, podmínky, kroky a očekávaný výsledek.
Účastník testu nesmí mít osobní zájem na jeho výsledku, musí být objektivní a nebát se cokoliv říct.
Přísedící musí na případné otázky účastníka odpovídat tak, aby nenarušil výsledky testu.

Účastník je seznámen s účelem scénáře a dále postupuje podle jeho kroků.
Během průchodu každého scénáře popisuje svou interakci s aplikací a své myšlenky, přísedící pořizuje poznámky či průběh testování nahrává.
Po skončení průchodu jsou poznámky zpracovány a výsledek je porovnán s očekávaným výsledkem scénáře.

Jednotlivé testovací scénáře reflektují stejnojmenné uživatelské případy ze sekce~\ref{sec:uzivatelskePripady}.

\paragraph{TC1 -- Vytvoření uživatelského účtu}

Účelem tohoto testovacího scénáře je vytvoření nového uživatelského účtu.
Před začátkem průchodu je aplikace ve stavu, kdy není přihlášen žádný uživatel a je zobrazen přihlašovací formulář.
Doba průchodu celého scénáře by neměla být delší než 5 minut.

Kroky scénáře:
\begin{enumerate}
    \item Účastník klikne na odkaz \texttt{Registrovat se}.
    \item Vyplní registrační formulář.
    \item Formulář odešle kliknutím na tlačítko \texttt{Registrovat se}.
\end{enumerate}

\pagebreak

Očekávaný výsledek po průchodu scénáře:
\begin{enumerate}
    \item Je vytvořen nový uživatelský účet se zadanými údaji.
    \item Na zadaný email je odeslán uvítací email.
    \item Vytvořený uživatel je přihlášen do aplikace.
    \item Je zobrazen prázdný seznam vytvořených dokumentů spolu s tlačítkem pro vytvoření nového dokumentu.
\end{enumerate}

\paragraph{TC2 -- Přihlášení do aplikace}

Účelem tohoto testovacího scénáře je přihlášení uživatele pomocí již vytvořeného uživatelského účtu.
Před začátkem průchodu je aplikace ve stavu, kdy není přihlášen žádný uživatel a je zobrazen přihlašovací formulář.
Doba průchodu celého scénáře by neměla být delší než 5 minut.

Kroky scénáře:
\begin{enumerate}
    \item Účastník vyplní přihlašovací formulář.
    \item Formulář odešle kliknutím na tlačítko \texttt{Přihlásit se}.
\end{enumerate}

Očekávaný výsledek po průchodu scénáře:
\begin{enumerate}
    \item Uživatel je přihlášen do aplikace.
    \item Je zobrazen seznam vytvořených dokumentů spolu s tlačítkem pro vytvoření nového dokumentu.
\end{enumerate}

\paragraph{TC3 -- Změna uživatelských údajů}

Účelem tohoto testovacího scénáře je změna uživatelských údajů přihlášeného uživatele.
Před začátkem průchodu je aplikace ve stavu, kdy je uživatel přihlášen a je zobrazen seznam vytvořených dokumentů.
Doba průchodu celého scénáře by neměla být delší než 5 minut.

Kroky scénáře:
\begin{enumerate}
    \item Účastník otevře vyskakovací menu s možnostmi kliknutím na ikonu uživatele.
    \item Pokračuje kliknutím na tlačítko \texttt{Nastavení}.
    \item Vyplní formulář pro změnu uživatelských údajů.
    \item Formulář odešle kliknutím na tlačítko \texttt{Uložit změny}.
\end{enumerate}

Očekávaný výsledek po průchodu scénáře:
\begin{enumerate}
    \item Uživatelovi údaje jsou změněny.
    \item Je zobrazeno plovoucí upozornění o výsledku akce.
\end{enumerate}

\paragraph{TC4 -- Změna výchozího nastavení dokumentů}

Účelem tohoto testovacího scénáře je změna výchozího nastavení pro nově vytvořené dokumenty.
Před začátkem průchodu je aplikace ve stavu, kdy je uživatel přihlášen a je zobrazen seznam vytvořených dokumentů.
Doba průchodu celého scénáře by neměla být delší než 10 minut.

Kroky scénáře:
\begin{enumerate}
    \item Účastník otevře vyskakovací menu s možnostmi kliknutím na ikonu uživatele.
    \item Pokračuje kliknutím na tlačítko \texttt{Nastavení}.
    \item Přepne se do požadovaného nastavení kliknutím na záložku \texttt{Nastavení dokumentů}.
    \item Vyplní formulář s výchozím nastavením pro nově vytvořené dokumenty.
    \item Formulář odešle kliknutím na tlačítko \texttt{Uložit změny}.
\end{enumerate}

Očekávaný výsledek po průchodu scénáře:
\begin{enumerate}
    \item Výchozí nastavení údaje pro nově vytvořené dokumenty je uloženo.
    \item Je zobrazeno plovoucí upozornění o výsledku akce.
\end{enumerate}


\paragraph{TC5 -- Zobrazení seznamu dokumentů}

Účelem tohoto testovacího scénáře je zobrazení seznamů vytvořených, sdílených a posledních dokumentů.
Před začátkem průchodu je aplikace ve stavu, kdy je uživatel přihlášen a je zobrazen seznam vytvořených dokumentů.
Doba průchodu celého scénáře by neměla být delší než 5 minut.

Kroky scénáře:
\begin{enumerate}
    \item Účastník otevře navigační menu kliknutím na ikonu navigace.
    \item Pokračuje kliknutím na tlačítka \texttt{Mé dokumenty}, \texttt{Sdílené} a \texttt{Poslední}.
\end{enumerate}

Očekávaný výsledek po průchodu scénáře:
\begin{enumerate}
    \item Postupně jsou zobrazeny všechny tři seznamy dokumentů.
\end{enumerate}


\paragraph{TC6 -- Vytvoření nového dokumentu}

Účelem tohoto testovacího scénáře je vytvoření nového dokumentu a nastavení jeho jména.
Před začátkem průchodu je aplikace ve stavu, kdy je uživatel přihlášen a je zobrazen seznam vytvořených dokumentů.
Doba průchodu celého scénáře by neměla být delší než 5 minut.

\pagebreak

Kroky scénáře:
\begin{enumerate}
    \item Účastník klikne na tlačítko \texttt{Vytvořit nový dokument}.
    \item Vyplní jméno dokumentu.
    \item Jméno uloží kliknutím na tlačítko s ikonou uložit.
\end{enumerate}

Očekávaný výsledek po průchodu scénáře:
\begin{enumerate}
    \item Nový dokument je vytvořen.
    \item Je zobrazen textový editor nově vytvořeného dokumentu.
    \item Dokument má nastavené vyplněné jméno.
\end{enumerate}

\paragraph{TC7 -- Otevření textového editoru}

Účelem tohoto testovacího scénáře je zobrazení textového editoru pro již existujícího sdíleného dokumentu.
Před začátkem průchodu je aplikace ve stavu, kdy je uživatel přihlášen a je zobrazen seznam vytvořených dokumentů.
Doba průchodu celého scénáře by neměla být delší než 5 minut.

Kroky scénáře:
\begin{enumerate}
    \item Účastník otevře navigační menu kliknutím na ikonu navigace.
    \item Zobrazí seznam sdílených dokumentů kliknutím na tlačítko \texttt{Sdílené}.
    \item Pokračuje kliknutím na název vybraného dokumentu v seznamu.
\end{enumerate}

Očekávaný výsledek po průchodu scénáře:
\begin{enumerate}
    \item Je zobrazen textový editor vybraného dokumentu.
\end{enumerate}

\paragraph{TC8 -- Změna nastavení dokumentu}

Účelem tohoto testovacího scénáře je změna nastavení již vytvořeného dokumentu.
Před začátkem průchodu je aplikace ve stavu, kdy je uživatel přihlášen a je zobrazen seznam vytvořených dokumentů.
Doba průchodu celého scénáře by neměla být delší než 10 minut.

Kroky scénáře:
\begin{enumerate}
    \item Účastník klikne na název vybraného dokumentu v seznamu.
    \item Pokračuje kliknutím na tlačítko nastavení dokumentu s ikonou ozubeného kola.
    \item Upraví nastavení dokumentu.
\end{enumerate}

Očekávaný výsledek po průchodu scénáře:
\begin{enumerate}
    \item Je zobrazen textový editor vybraného dokumentu a otevřeno postranní menu s jeho nastavením.
    \item Změny nastavení byly ve skutečném čase uloženy a aplikovány na otevřený editor.
\end{enumerate}

\paragraph{TC9 -- Přizvání uživatele k dokumentu}

Účelem tohoto testovacího scénáře je přizvání dalšího uživatel k editaci již vytvořeného dokumentu.
Před začátkem průchodu je aplikace ve stavu, kdy je uživatel přihlášen a je zobrazen seznam vytvořených dokumentů.
Doba průchodu celého scénáře by neměla být delší než 10 minut.

Kroky scénáře:
\begin{enumerate}
    \item Účastník klikne na název vybraného dokumentu v seznamu.
    \item Pokračuje kliknutím na tlačítko \texttt{Sdílet}.
    \item Zobrazí formulář pro vytvoření nové pozvánky kliknutím na tlačítko \texttt{Nová pozvánka}.
    \item Vyplní formulář pro vytvoření nové pozvánky.
    \item Formulář odešle kliknutím na tlačítko \texttt{Pozvat}.
\end{enumerate}

Očekávaný výsledek po průchodu scénáře:
\begin{enumerate}
    \item Je zobrazen textový editor vybraného dokumentu a otevřeno postranní menu s možnostmi jeho sdílení.
    \item Pozvaný uživatel vidí dokument ve svém seznamu sdílených dokumentů.
    \item Pozvaný uživatel se zobrazil v seznamu pozvaných uživatel v otevřeném postranním menu.
\end{enumerate}

\paragraph{TC10 -- Komunikační vlákno dokumentu}

Účelem tohoto testovacího scénáře je zobrazení komunikačního vlákna již vytvořeného dokumentu a přidání nové zprávy.
Před začátkem průchodu je aplikace ve stavu, kdy je uživatel přihlášen a je zobrazen seznam vytvořených dokumentů.
Doba průchodu celého scénáře by neměla být delší než 5 minut.

Kroky scénáře:
\begin{enumerate}
    \item Účastník klikne na název vybraného dokumentu v seznamu.
    \item Pokračuje kliknutím na tlačítko komunikačního vlákna s ikonou konzervační bubliny.
    \item Vyplní pole pro text nové zprávy.
    \item Zprávu odešle kliknutím na tlačítko s ikonou odeslat.
\end{enumerate}

Očekávaný výsledek po průchodu scénáře:
\begin{enumerate}
    \item Je zobrazen textový editor vybraného dokumentu a otevřeno postranní menu s komunikačním vláknem dokumentu.
    \item Odeslaná zpráva se zobrazila v komunikačním vláknem dokumentu.
\end{enumerate}


\subsection{Účastníci testování}\label{subsec:účastníciTestování}

Účastníci jsou nejdůležitější částí uživatelského testování.
Procházejí jednotlivé testovací scénáře a rozhodují tak o přístupnosti uživatelského rozhraní aplikace.

Počet účastníků testování je dán především velikostí testované aplikací a výsledky, kterých chceme dosáhnout.
Nízký počet účastníků umožňuje detailnější popis jednotlivých průchodů.
Vyšší počet účastníků naopak dosahuje výsledků statistické povahy nápomocné při rozhodování mezi různými způsoby implementace.

Uživatelského testování implementovaného prototypu aplikace se zúčastnili 3 účastníci.
Pro ochranu jejich soukromí jsou však tito účastníci anonymizováni.
Každý z účastníků prošel všemi testovacími scénáři definovanými v sekci~\ref{subsec:testovacíScénáře}.

\paragraph{Účastník 1}
Student \acrshort{ČVUT}, 22 let, jeho koníčkem je programování a elektronika.
Ovládá pokročilé počítačové programy na vysoké úrovni a na internetu je jako doma.
S nástroji pro kolaborativní editaci se setkává běžně v každodenním životě.

\paragraph{Účastník 2}
Studentka Univerzity Jana Evangelisty Purkyně v Ústí nad Labem, 20 let, mezi její koníčky patří cyklistika a psy.
Ovládá běžné kancelářské programy na základní úrovni a běžně se pohybuje po internetu.
S nástroji pro kolaborativní editaci textů se již setkala, ale nikdy je aktivně nevyužívala.

\paragraph{Účastník 3}
Zaměstnankyně Městského úřadu v Ústí nad Labem, 42 let, mezi její koníčky patří Jóga a práce na zahradě.
Ovládá běžné kancelářské programy na vysoké úrovni, avšak na internetu používá pouze základní služby.
S nástroji pro kolaborativní editaci se nikdy nesetkala.

\subsection{Výsledky testování}\label{subsec:výsledkyTestování}

V této části jsou popsané výsledky jednotlivých průchodů s účastníky uživatelského testování.
Každý účastník hodnotil průchod testovacím scénářem hodnotou 1--5, kde 5 znamená intuitivní rozhraní (dokázali ho splnit sami a bez přemýšlení) a 1 neintuitivní (bez pomoci přísedícího nedokázali scénář dokončit).

Účastníci byli před začátkem testování seznámeni s prototypem aplikace, účelem každého scénářem a způsobem jejich hodnocení.
Před průchodem každého scénáře byl prototyp aplikace nastaven do stavu podle jeho počátečních podmínek.
Jednotlivé kroky scénáře byli účastníkům odhaleny pouze v případě potřeby jako nápověda pro dokončení průchodu.

\paragraph{Průchody účastníka 1}

Účastník odhalil problém při průchodu testovacím scénářem Změna uživatelských údajů (TC3).
Nevšiml si pole pro potvrzení změn aktuálním heslem a formulář se mu na první pokus nepodařilo odeslat.
Po zobrazení chyby pole vyplnil a scénář úspěšně dokončil.
Podle účastníka je pole nevhodně umístěno a působí dojem, že je určeno pouze pro potvrzení při vyplnění ostatních polí pro změnu hesla.

Účastník prošel všemi ostatními testovacími scénáři bez odhalení dalších problémů uživatelského rozhraní.
Podle účastníka se uživatelské rozhraní aplikace řídí zavedenými zvyklostmi a je jednoduché se v něm orientovat.

\begin{table}[ht!]
    \centering
    \caption{Hodnocení scénářů po průchodu účastníkem 1}
    \label{tab:poPrůchoduÚčastníkem1}
    \begin{tabular}{r|c}
        & Hodnocení (1--5) \\ \hline
        TC1 & 5 \\
        TC2 & 5 \\
        TC3 & 3 \\
        TC4 & 5 \\
        TC5 & 5 \\
        TC6 & 5 \\
        TC7 & 5 \\
        TC8 & 5 \\
        TC9 & 5 \\
        TC10 & 5 \\
    \end{tabular}
\end{table}


\paragraph{Průchody účastníka 2}

Účastník při průchodu testovacími scénáři Vytvoření uživatelského účtu (TC1), Přihlášení do aplikace (TC2), Změna výchozího nastavení dokumentů (TC4), Zobrazení seznamu dokumentů (TC5),  Otevření textového editoru (TC7), Změna nastavení dokumentu (TC8), Přizvání uživatele k dokumentu (TC9) a Komunikační vlákno dokumentu (TC10) neodhalil žádný problém.
Testované uživatelské rozhraní bylo pro účastníka známe z ostatních aplikací společnosti Google (díky použitému Material Designu).

Při průchodu testovacím scénářem Změna uživatelských údajů (TC3) byl odhalen problém s polem aktuální heslo.
Toto pole účastním nevyplnil a při první pokus o vyplnění skončil chybou.
Po zobrazení a přečtení chyby účastník pole zpětně vyplnil a scénář dokončil.
Pole pro zadaní aktuálního hesla není podle slov účastníka dostatečně označené a není na první pohled poznat, že je povinné.

Na další a poslední problém účastník odhalil při průchodu testovacím scénářem Vytvoření nového dokumentu (TC6).
Účastník neuložil název dokumentu kliknutím na tlačítko uložit a považoval scénář za dokončený.
Podle slov účastníka ho zmátla ikona stavu synchronizace s textem \enquote{Všechny změny uloženy} a předpokládal, že se tato informace vztahuje i k názvu dokumentu.

Účastník potvrdil existenci problému týkající se umístění pole s aktuálním heslem v uživatelském nastavení.
Ale také upozornil na další problém, který by mohl negativně postihnout zkušenost ostatních uživatelů s prototypem aplikace.

\begin{table}[ht!]
    \centering
    \caption{Hodnocení scénářů po průchodu účastníkem 2}
    \label{tab:poPrůchoduÚčastníkem2}
    \begin{tabular}{r|c}
        & Hodnocení (1--5) \\ \hline
        TC1 & 5 \\
        TC2 & 5 \\
        TC3 & 4 \\
        TC4 & 5 \\
        TC5 & 5 \\
        TC6 & 3 \\
        TC7 & 5 \\
        TC8 & 5 \\
        TC9 & 5 \\
        TC10 & 5 \\
    \end{tabular}
\end{table}

\paragraph{Průchody účastníka 3}

Účastník prošel bez problémů testovacími scénáři Vytvoření uživatelského účtu (TC1), Přihlášení do aplikace (TC2), Změna výchozího nastavení dokumentů (TC4), Změna nastavení dokumentu (TC8) a Přizvání uživatele k dokumentu (TC9).
Uživatelské rozhraní v rámci těchto scénářů je podle účastníka navrženo dle zvyklostí a je na první pohled intuitivní.

Drobné problémy byly odhaleny účastníkovým průchodem testovacích scénářů Vytvoření nového dokumentu (TC6) a Komunikační vlákno dokumentu (TC10).
V prvním případě si účastník nevšiml tlačítka pro uložení jména dokumentu a pokračoval bez uložení, protože očekával automatické ukládání názvu dokumentu.
V druhém případě účastním nepoznal ikonu pro otevření komunikačního vlákna a odeslání nové zprávy, ale ke správnému výsledku ho dovedly vyskakovací popisy jednotlivých tlačítek.

První závažný problém nastal při průchodu testovacím scénářem Změna uživatelských údajů (TC3).
Účastník nevyplnil pole pro aktuální heslo a formulář se mu tak nepodařilo odeslat.
Podle účastníka je pole pro zadání aktuálního hesla nedostatečně označené a svým umístěním spadá pouze pod změnu přihlašovacího hesla.

Účastníkův průchod testovacím scénářem Zobrazení seznamu dokumentů (TC5) odhalil další problém.
Účastník nepoznal ikonu pro otevření bočního menu aplikace a měl tendenci náhodně klikat na všechna tlačítka za účelem jeho nalezení.
Ikona podle účastníka není dobře označena a boční menu očekával stále otevřené.

Na poslední problém účastník narazil při průchodu testovacího scénáře Otevření textového editoru (TC7) a opět se jednalo o problém s bočním menu aplikace.
Účastník chtěl poprvé zobrazit seznam s ním sdílených dokumentů pomocí tlačítka pro řazení seznamu dokumentů, ale po jeho aktivaci si ihned uvědomil účel tohoto tlačítka a pokračoval ve scénáři.

Některé problémy uživatelského rozhraní odhalené účastníkem souvisí s jeho omezenou znalostí běžně zavedených prvků uživatelského rozhraní na internetu (například neznalost ikony pro menu či odeslání zprávy).
Poukázal ovšem i na závažnější chyby uživatelského rozhraní, kterými je vhodné se do budoucna zabývat.

\begin{table}[ht!]
    \centering
    \caption{Hodnocení scénářů po průchodu účastníkem 3}
    \label{tab:poPrůchoduÚčastníkem3}
    \begin{tabular}{r|c}
        & Hodnocení (1--5) \\ \hline
        TC1 & 5 \\
        TC2 & 5 \\
        TC3 & 1 \\
        TC4 & 5 \\
        TC5 & 2 \\
        TC6 & 4 \\
        TC7 & 3 \\
        TC8 & 5 \\
        TC9 & 5 \\
        TC10 & 4 \\
    \end{tabular}
\end{table}

\subsubsection{Shrnutí uživatelského testování}

Pomocí uživatelského testovaní byly identifikovány problematické prvky uživatelského rozhraní implementovaného prototypu.
Mezi závažné problémy jsem s ohledem na jejich odhalení více účastníky zařadil pole aktuálního hesla ve formuláře pro změnu uživatelským nastavením a nutnost potvrzení nového jména dokumentu namísto jeho automatického uložení.

Tyto problémy nejsou překážkou v používání prototypu jako takového, ale mohou být důvodem nepříjemné uživatelské zkušenosti, a proto je nutné se jimi do budoucna zabývat.
Návrh bezchybného uživatelského rozhraní nebyl cílem této práce, a proto tyto nalezené problémy nepovažuji za jeho nesplnění.
