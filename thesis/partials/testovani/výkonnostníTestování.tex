% !TEX encoding = UTF-8 Unicode
% -*- coding: UTF-8; -*-
% vim: set fenc=utf-8
%\inputencoding{utf8}

\section{Výkonnostní testování}\label{sec:systémovéTestování}

Výkon implementovaného prototypu aplikace a komponentu editoru jsem testoval při rozdílných počtech připojených uživatelů.
Aplikace byla schopna synchronizovat text jednoho dokumentu pro přes 100 připojených uživatelů a to bez znatelného zpomalení.
Limitujícím faktorem počtu uživatelů byl výkon počítače, který simuloval jednotlivé připojené klienty.

Měření probíhalo otevřením daného počtu aktivních oken editoru na jiném než pozorovaném počítači (z důvodu ovlivnění výsledků mezi klienty).
Na pozorovaném počítači byl měřen čas odchozí a příchozí operace mezi dvěma pozorovanými editory.
Čas průměrné odezvy mezi testovaným počítačem a testovaným aplikačním serverem byl 60 ms.

Jak je z výsledků v tabulce~\ref{tab:vysledkyVýkonostníhoTestování} patrné, tak server byl i s tímto počtem připojených schopen propagovat operace téměř okamžitě.
Ani při počtu 100 připojených klientů nedocházelo k výraznějšímu zpoždění propagace operací.
Rozdíly mezi jednotlivými testy jsou v rámci chyby měření, která mohla být způsobena bezdrátovým připojením k internetu v době testovaní.

\begin{table}[ht!]
    \centering
    \caption{Výsledky testování doby synchronizace}
    \label{tab:vysledkyVýkonostníhoTestování}
    \begin{tabular}{r|ccc}
        & Průměr & Nejlepší & Nejhorší \\ \hline
        Pouze pozorovaní uživatelé & 91 ms & 65 ms & 157 ms \\
        10 připojených uživatelů & 109 ms & 62 ms & 249 ms \\
        50 připojených uživatelů & 117 ms & 60 ms & 225 ms \\
        100 připojených uživatelů & 113 ms & 59 ms & 264 ms
    \end{tabular}
\end{table}

Tento výsledek byl očekávaný a je dán povahou použitého synchronizačního algoritmu~\gls{OT} (více k algoritmu v sekci~\ref{subsec:operacniTransformace}).
Server propagovanou operaci před odesláním pouze transformuje oproti souběžným operacím a aplikuje na vlastní kopii textu daného dokumentu.
Každý jednotlivý klient je zodpovědný za správnou transformaci všech přijatých operací a jejich aplikaci na vlastní kopii textu, ale také za odchycení uživatelského vstupu a jeho převod na soubor operací.

Toto je dobrá zpráva, protože právě server a databáze je dnes, i přes široké možnosti škálování, stále úzkým hrdlem většiny aplikací.
Výpočet u uživatele samotného lze dále optimalizovat, ale jeho náročnost se neprojevuje u ostatních uživatelů aplikace tak, jako kdyby probíhal na serveru.
