% !TEX encoding = UTF-8 Unicode
% -*- coding: UTF-8; -*-
% vim: set fenc=utf-8
%\inputencoding{utf8}

\section{Výkonnostní testování}\label{sec:systémovéTestování}

Výkon implementovaného prototypu aplikace a komponentu editoru jsem testoval při rozdílných počtech připojených uživatelů.
Aplikace byla schopna synchronizovat text jednoho dokumentu pro přes 100 připojených uživatelů a to bez znatelného zpomalení.
Limitujícím faktorem počtu uživatelů byl výkon počítače, který simuloval jednotlivé připojené klienty.

Tento výsledek byl očekávaný a je dán povahou použitého synchronizačního algoritmu~\gls{OT} (více k algoritmu v sekci~\ref{subsec:operacniTransformace}).
Server propagovanou operaci před odesláním pouze transformuje oproti souběžným operacím a aplikuje na vlastní kopii textu daného dokumentu.
Každý jednotlivý klient je zodpovědný za správnou transformaci všech přijatých operací a jejich aplikaci na vlastní kopii textu, ale také za odchycení uživatelského vstupu a jeho převod do na Operace.

Toto je dobrá zpráva, protože právě server a databáze je dnes, i přes široké možnosti škálování, stále úzkým hrdlem většiny aplikací.
Výpočet u uživatele samotného lze dále optimalizovat, ale jeho náročnost se neprojevuje u ostatních uživatelů aplikace tak, jako kdyby probíhal na serveru.

