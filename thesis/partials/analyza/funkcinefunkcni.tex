
\section{Seznam funkčních požadavků}\label{sec:funcniPozadavky}
Před začátkem návrh jakékoli aplikace je důležité si definovat funkční požadavky.

% TODO: rozepsat jednotlivé funkční požadavky
Jednotlivé funkční požadavky bych chtěl rozepsat jednou, či dvěma větami.
Ale musím vymyslet jak to v latexu udělat, protože podseznam je ošklivý.

\subsection{Správa uživatelů}\label{subsec:správaUživatelů}

Uživatel se bude moci v rámci systému:
\begin{enumerate}[label=F1.\arabic*.]
    \item zaregistrovat pomocí uživatelského jména a hesla,
    \item přihlásit pomocí údajů uvedených při registraci,
    \item provést změnu svých přihlašovací údajů,
    \item změnit výchozí nastavení pro nově vytvořené dokumenty a
    \item v případě zapomenutí svého přístupového hesla použít formulář k jeho obnově.
\end{enumerate}

\subsection{Správa dokumentů}\label{subsec:správaDokumentů}

V systému půjdou s dokumenty provést následující akce:
\begin{enumerate}[label=F2.\arabic*.]
    \item vytvoření dokumentu přihlášeným uživatelem,
    \item odstranění dokumentu jeho majitelem,
    \item zobrazení všech uživatelových dokumentů,
    \item zobrazení naposledy otevřených dokumentů,
    \item změna nastavení vzhledu dokumentu a vlastností jeho editoru,
    \item přizvání uživatele k editaci dokumentu nebo k jeho náhledu pomocí jeho uživatelského jména,
    \item vytvoření veřejného odkazu dokumentu pro přizvání uživatele bez vytvořeného účtu,
    \item editace dokumentu ve skutečném čase včetně barevně označených kurzorů ostatních uživatel upravující dokument a
    \item diskutovat o dokumentu ve skutečném čase s ostatními uživateli upravující dokument.
\end{enumerate}

\section{Nefunkční požadavky}\label{sec:nefuncniPozadavky}

Pro potřeby projektu Laplace-IDE byli vyhrazeny následující nefunkční požadavky:
\begin{enumerate}[label=N\arabic*.]
    \item validní kód HTML5 (více o jazyce HTML5 v části~\ref{subsec:html5}) a \gls{CSS3},
    \item programovací jazyk Javascript (více o jazyce JavaScript v části~\ref{subsec:javascript)},
    \item prostředí Node.js (více o prostředí Node.js v části~\ref{subsec:nodejs}) pro server a
    \item použití knihovny React (více o knihovně React v části~\ref{subsec:reactjs}) k implementaci komponenty Editoru.
\end{enumerate}
