
\section{Seznam funkčních požadavků}\label{sec:funcniPozadavky}
Před začátkem návrh jakékoli aplikace je důležité definovat funkční požadavky.
Funkční požadavky slouží k vymezení toho, co má aplikace umět a co už umět nemusí.

\subsection{Správa uživatelů}\label{subsec:správaUživatelů}

Uživatel se může v rámci aplikace zaregistrovat pomocí uživatelského jména a hesla.
Uživatelské jméno je pro každého uživatele unikátní a slouží jako jeho identifikátor mezi ostatními uživateli.
Pomocí údajů uživatelské jména a heslo, které uvedených při registraci, se může uživatel následně do aplikace přihlásit.

Přihlášený uživatel může své údaje kdykoliv změnit.
Dále může kdykoliv změnit vzhled a vlastnosti editoru pro nově vytvořené dokumenty.
Toto nastavení však nemá žádný vliv na dokumenty již vytvořené.

V případě, že uživatel zapomene své přístupové heslo, může využít formuláře pro obnovu přístupového hesla.
Po odeslání tohoto formuláře aplikace odešle na emailovou adresu, kterou uživatel zadal při registraci, postup, s jehož pomocí si uživatel může nastavit nové heslo pro přístup do aplikace.

% \begin{enumerate}[label=F1.\arabic*.]
%     \item zaregistrovat pomocí uživatelského jména a hesla,
%     \item přihlásit pomocí údajů uvedených při registraci,
%     \item provést změnu svých přihlašovací údajů,
%     \item změnit výchozí nastavení pro nově vytvořené dokumenty a
%     \item v případě zapomenutí svého přístupového hesla použít formulář k jeho obnově.
% \end{enumerate}

\subsection{Správa dokumentů}\label{subsec:správaDokumentů}

Přihlášený uživatel můžu v aplikaci vytvořit nový dokument.
Libovolný jím vytvořený dokument může uživatel také odstranit.
Uživatel, pokud k tomu má dostatečné oprávnění, může změnit vzhled a vlastnosti editoru jednotlivých dokumentů.

Uživatel po přihlášení uvidí seznam jím vytvořených dokumentů.
Dále si přihlášený uživatel může zobrazit seznam naposledy otevřených dokumentů a seznam dokumentů, ke kterým byl uživatel někým přizván.

Každý dokument má jednoznačně určený veřejný odkaz, který lze sdílet mezi uživateli.
Uživatel s dostatečným oprávnění může změnit oprávnění veřejného odkazu.
Toto oprávnění získají všichni uživatelé, kteří mají přístup k dokumentu (znají jeho veřejný odkaz).
Uživatel s dostatečným oprávněním může přizvat dalšího uživatele ke spolupráci na dokumentu.

Uživatel s dostatečným oprávněním může zobrazit obsah dokumentu a v případě dokument ve skutečném čase editovat.
Jednotlivý uživatelé upravující obsah dokumentu jsou od sebe barevně odlišeni.
Jejich kurzory (či vybraný text) jsou viditelné ostatními uživateli a označeni stejnou barvou.

Uživatel s dostatečným oprávněním může zobrazit diskuzní vlákno dokumentu a pod je přihlášený může do něj i přispívat.
Diskuzní vlákno je aktualizováno ve skutečném čase, tedy uživatel vidí nové zprávy bez nutnosti jakkoliv se stránkou interagovat.
Uživatelovi zprávy v diskuzním vláknu jsou označeny stejnou barvou jako jeho kurzor v editoru dokumentu.

% \begin{enumerate}[label=F2.\arabic*.]
%     \item vytvoření dokumentu přihlášeným uživatelem,
%     \item odstranění dokumentu jeho majitelem,
%     \item zobrazení všech uživatelových dokumentů,
%     \item zobrazení naposledy otevřených dokumentů,
%     \item změna nastavení vzhledu dokumentu a vlastností jeho editoru,
%     \item přizvání uživatele k editaci dokumentu nebo k jeho náhledu pomocí jeho uživatelského jména,
%     \item vytvoření veřejného odkazu dokumentu pro přizvání uživatele bez vytvořeného účtu,
%     \item editace dokumentu ve skutečném čase včetně barevně označených kurzorů ostatních uživatel upravující dokument a
%     \item diskutovat o dokumentu ve skutečném čase s ostatními uživateli upravující dokument.
% \end{enumerate}

\section{Seznam nefunkčních požadavků}\label{sec:nefuncniPozadavky}

Pro potřeby projektu Laplace-IDE byly vyhrazeny následující nefunkční požadavky:
\begin{enumerate}
    \item validní kód HTML5 (více o jazyce HTML5 v sekci~\ref{subsec:html5}) a \gls{CSS3},
    \item programovací jazyk Javascript (více o jazyce JavaScript v sekci~\ref{subsec:javascript)},
    \item prostředí Node.js (více o prostředí Node.js v sekci~\ref{subsec:nodejs}) pro server a
    \item použití knihovny React (více o knihovně React v čsekciásti~\ref{subsec:reactjs}) k implementaci komponenty Editoru.
\end{enumerate}
