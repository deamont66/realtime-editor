% !TEX encoding = UTF-8 Unicode
% -*- coding: UTF-8; -*-
% vim: set fenc=utf-8

\section{Technologie}\label{sec:technologie}
V této kapitole popisuji použité technologie vycházející z nefunkčních požadavků, které jsou uvedeny v kapitole~\ref{sec:nefuncniPozadavky}.

\subsection{HTML5}\label{subsec:html5}

HTML5 je značkovací jazyk používaný pro reprezentaci a strukturování  obsahu na internetu (přesněji na WWW\footnote{World Wide Web}).
Jedná se již o pátou verzi standardu HTML\footnote{Hypertext Markup Language} (doporučená verze podle W3C\footnote{The World Wide Web Consortium} v roce 2018 je HTML 5.2~\cite{w3c:html52}).
Tato verze do standartu přidává mimo jiné nové elementy, atributy a funkcionalitu.
Pod pojem HTML5 také často zařazujeme rozsáhlou množinu moderních technologií, které umožňují tvorbu více rozmanitých a mocných webových stránek a aplikací.~\cite{mozzila:html5}

HTML vytvořil Tim Berners-Lee a HTML standart byl definován ve spolupráci s organizací IETF\footnote{Internet Engineering Task Force} v roce 1993~\cite{html:autor}.
Od roku 1996 převzala vývoj HTML standartu organizace W3C~\cite{w3c:html32}.
Roku 2008 se k W3C přidala organizace WHATWG\footnote{Web Hypertext Application Technology Working Group} a započali vývoj standartu HTML5, který společně vydali v roce 2014~\cite{w3c:html5}.


\subsection{JavaScript}\label{subsec:javascript}

JavaScript pod pracovním názvem LiveScript vytvořil Brendan Eich v roce 1995, kdy působil jako inženýr ve firmě Netscape.
Přejmenování na JavaScript bylo marketingové rozhodnutí a mělo využít tehdejší popularity programovacího jazyka Java od Sun Microsystem a to přesto, že tyto jazyky spolu téměř nesouvisí.
Javascript byl poprvé vydán jako součást prohlížeče Netscape 2 roku 1996.
Později téhož roku představila firma Microsoft pro svůj prohlížeč Internet Explorer 3 jazyk JScript, který byl JavaScriptu velice podobný.~\cite{mozzila:javascript}

Roku 1997 vydala organizace ECMA\footnote{European Computer Manufacturer's Association} první verzi standartu ECMAScript, který z původního JavaScriptu a JScriptu vycházel~\cite{ecma:ecmascript1}.
Tento standart prošel v roce 1999 rozsáhlou aktualizací jako ECMAScript 3, která je bez větších změn používána dodnes~\cite{mozzila:javascript}.

Další verze ECMAScript standartu podle~\cite{mozzila:javascriptVerisions} jsou:
\begin{itemize}
    \item ECMAScript 5 z roku 2009,
    \item ECMAScript 5.1 z roku 2011 (\href{http://www.iso.org/iso/iso_catalogue/catalogue_tc/catalogue_detail.htm?csnumber=55755}{ISO/IEC 16262:2011}),
    \item ECMAScript 2015,
    \item ECMAScript 2016,
    \item ECMAScript 2017 a
    \item připravovaný standart ECMAScript 2018.
\end{itemize}

Javascript je navržen k běhu jako skriptovací jazyk v hostitelském prostředí, které musí zajistit mechanismy pro komunikaci mimo toho prostředí.
Nejčastějším hostitelským prostředím je webový prohlížeč, ale Javascript můžeme nalézt na místech jako jsou Adobe Acrobat, Adobe Photoshop, SVG vektorová grafika, serverové prostředí jako je NodeJS, NoSQL databáze jako je Apache CouchDB, vestavěné systémy a další.~\cite{mozzila:javascript}

Javascript je více paradigmový, dynamický jazyk s datovými typy a operátory, standardními vestavěnými objekty a metodami.
Jeho syntaxe je založena na jazycích Java a C.
JavaScript podporuje objektově orientované programování pomocí objektových prototypů namísto tříd jako je tomu například u jazyku Java.
Dále také podporuje principy funkcionální programování -- funkce jsou také objekty.~\cite{mozzila:javascript}

\subsection{Node.js}\label{subsec:nodejs}

Node.js je JavaScriptové běhové prostředí (anglicky runtime environment), které používá událostmi řízenou architekturu umožňující asynchronní přístup k I/O\footnote{Vstupní/výstupní} operacím.
Tato architektura umožňuje optimalizovat propustnost a škálovatelnost webových aplikací s mnoha I/O operacemi, ale také webových aplikací ve skutečném čase (například komunikační programy nebo hry v prohlížeči).~\cite{node:article2013}

Toto je v kontrastu s dnešními více známými modely souběžnosti, kde se využívají vlákna operačního systému.
Síťová komunikace založená na vláknech je relativně neefektivní a její použití je velmi složité.~\cite{node:about}

Node.js využívá V8 JavaScript interpret vytvořený společností Google pod skupinou The Chromium Project pro prohlížeč Google Chrome a ostatní prohlížeče postavené na Chromium~\cite{node:about}.
V8 je napsaný v C++ a kompiluje JavaScriptový kód do nativního strojového kódu namísto interpretace kódu za běhu programu.
Toto umožňuje vytvořit rychlé běhové prostředí, které nemusí čekat na překlad potřebného kódu.~\cite{node:article2013}

\subsection{ReactJS}\label{subsec:reactjs}
Javascript je.

\subsection{MongoDB}\label{subsec:mongodb}
Javascript je 2.