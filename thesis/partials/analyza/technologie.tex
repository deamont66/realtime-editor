% !TEX encoding = UTF-8 Unicode
% -*- coding: UTF-8; -*-
% vim: set fenc=utf-8

\section{Použité technologie}\label{sec:technologie}
V této část popisuji použité technologie vycházející z nefunkčních požadavků, které jsou uvedeny v kapitole~\ref{sec:nefuncniPozadavky}.

\subsection{HTML5}\label{subsec:html5}

HTML5 je značkovací jazyk používaný pro reprezentaci a strukturování  obsahu na internetu (přesněji na WWW\footnote{World Wide Web}).
Jedná se již o pátou verzi standardu HTML\footnote{Hypertext Markup Language} (doporučená verze podle W3C\footnote{The World Wide Web Consortium} v roce 2018 je HTML 5.2~\cite{w3c:html52}).
Tato verze do standartu přidává mimo jiné nové elementy, atributy a funkcionalitu.
Pod pojem HTML5 také často zařazujeme rozsáhlou množinu moderních technologií, které umožňují tvorbu více rozmanitých a mocných webových stránek a aplikací.~\cite{mozzila:html5}

HTML vytvořil Tim Berners-Lee a HTML standart byl definován ve spolupráci s organizací IETF\footnote{Internet Engineering Task Force} v roce 1993~\cite{html:autor}.
Od roku 1996 převzala vývoj HTML standartu organizace W3C~\cite{w3c:html32}.
Roku 2008 se k W3C přidala organizace WHATWG\footnote{Web Hypertext Application Technology Working Group} a započali vývoj standartu HTML5, který společně vydali v roce 2014~\cite{w3c:html5}.


\subsection{JavaScript}\label{subsec:javascript}

JavaScript pod pracovním názvem LiveScript vytvořil Brendan Eich v roce 1995, kdy působil jako inženýr ve firmě Netscape.
Přejmenování na JavaScript bylo marketingové rozhodnutí a mělo využít tehdejší popularity programovacího jazyka Java od Sun Microsystem a to přesto, že tyto jazyky spolu téměř nesouvisí.
Javascript byl poprvé vydán jako součást prohlížeče Netscape 2 roku 1996.
Později téhož roku představila firma Microsoft pro svůj prohlížeč Internet Explorer 3 jazyk JScript, který byl JavaScriptu velice podobný.~\cite{mozzila:javascript}

Roku 1997 vydala organizace ECMA\footnote{European Computer Manufacturer's Association} první verzi standartu ECMAScript, který z původního JavaScriptu a JScriptu vycházel~\cite{ecma:ecmascript1}.
Tento standart prošel v roce 1999 rozsáhlou aktualizací jako ECMAScript 3, která je bez větších změn používána dodnes~\cite{mozzila:javascript}.

Další verze ECMAScript standartu podle~\cite{mozzila:javascriptVerisions} jsou:
\begin{itemize}
    \item ECMAScript 5 z roku 2009,
    \item ECMAScript 5.1 z roku 2011 (\href{http://www.iso.org/iso/iso_catalogue/catalogue_tc/catalogue_detail.htm?csnumber=55755}{ISO/IEC 16262:2011}),
    \item ECMAScript 2015,
    \item ECMAScript 2016,
    \item ECMAScript 2017 a
    \item připravovaný standart ECMAScript 2018.
\end{itemize}

Dnes pod označením JavaScript běžně chápeme právě standardizovaný ECMAScript~\cite{mozzila:javascript} a i já ho tak budu dále používat.

Javascript je navržen k běhu jako skriptovací jazyk v hostitelském prostředí, které musí zajistit mechanismy pro komunikaci mimo toho prostředí.
Nejčastějším hostitelským prostředím je webový prohlížeč, ale Javascript můžeme nalézt i na místech jako je Adobe Acrobat, Adobe Photoshop, SVG vektorová grafika, serverové prostředí NodeJS, NoSQL databáze Apache CouchDB, nejrůznější vestavěné systémy a další.~\cite{mozzila:javascript}

Javascript je více paradigmový, dynamický jazyk s datovými typy, operátory, standardními vestavěnými objekty a metodami.
Jeho syntaxe je založena na jazycích Java a C.
JavaScript podporuje objektově orientované programování pomocí objektových prototypů namísto tříd jako je tomu například u jazyku Java.
Dále také podporuje principy funkcionální programování -- funkce jsou také objekty.~\cite{mozzila:javascript}

\subsection{Node.js}\label{subsec:nodejs}

Node.js je JavaScriptové běhové prostředí (anglicky runtime environment), které používá událostmi řízenou architekturu umožňující asynchronní přístup k I/O\footnote{Vstupní/výstupní} operacím.
Tato architektura umožňuje optimalizovat propustnost a škálovatelnost webových aplikací s mnoha I/O operacemi, ale také webových aplikací ve skutečném čase (například komunikační programy nebo hry v prohlížeči).~\cite{node:article2013}

Toto je v kontrastu s dnešními více známými modely souběžnosti, kde se využívají vlákna operačního systému.
Síťová komunikace založená na vláknech je relativně neefektivní a její použití bývá velmi obtížné.~\cite{node:about}

Node.js využívá V8 JavaScript interpret vytvořený společností Google pod skupinou The Chromium Project pro prohlížeč Google Chrome a ostatní prohlížeče postavené na Chromium\footnote{Prohlížeč s otevřeným zdrojovým kódem od společnosti Google}~\cite{node:es6}.
V8, který je napsaný v C++, kompiluje JavaScriptový kód do nativního strojového kódu namísto jeho interpretace až za běhu programu.
To umožňuje vytvořit rychlé běhové prostředí, které nemusí čekat na překlad potřebného kódu.~\cite{node:article2013}

\subsection{React}\label{subsec:reactjs}

React je Javascriptová knihovna pro tvorbu uživatelského rozhraní~\cite{react:about}.
React byl vytvořen Jordanem Walkem, inženýrem ve společnosti Facebook, a byl poprvé použit v roce 2011.
Původně byl React určen výhradně pro použití na Facebook Timeline, ale Facebook inženýr Pete Hunt se rozhodl React použít i v aplikaci Instagram.
Postupně tak React zbavil závislostí na kód Facebooku a tím napomohl vzniku oficiální React knihovny.
React byl představen veřejnosti jako knihovna s otevřeným zdrojovým kódem v květnu roku 2013.~\cite{react:author}

Základním prvkem Reactu jsou takzvané komponenty, které přijímají neměnné vlastnosti a mohou definovat vlastní stavové proměnné.
Na základě těchto vlastností a stavu, pak komponenta může rozhodnout co bude jejích výstupem pro uživatele (pomocí metody render).
Tato vlastnost se nazývá jednosměrný datový tok (anglicky One-way data flow) a architekturu, kterou React implementuje, nazýváme Flux (ta je součástí Reactu už od samého počátku).~\cite{react:about}
Existují však komunitou vytvořené alternativní nástroje, které řeší datový tok v aplikaci, jako je například knihovna Redux~\cite{react:redux}.

\subsection{MongoDB}\label{subsec:mongodb}
MongoDB je nerelační databáze s otevřeným zdrojovým kódem vyvíjena společností MongoDB, Inc.
Díky zjednodušené reprezentaci dat pomocí dokumentů a jejich rychlému mapování na JSON\footnote{JavaScript Object Notation} není potřeba využívat složité ORM\footnote{Object-relational mapping} nástroje pro mapování dat na objekty, jako tomu je například u SQL\footnote{Structured Query Language} databází, a tím umožňuje urychlit celkový vývoj aplikací.~\cite{mongo:about}

Pro MongoDB jsem se rozhodl převážně s ohledem na jeho rychlost, která se u aplikace pro editací textů ve skutečném času hodí, ale také pro jeho rozsáhlou komunitu mezi vývojáři~\cite{mongo:speed,mongo:popularity}.
