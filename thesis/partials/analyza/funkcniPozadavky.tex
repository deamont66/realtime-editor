
\section{Seznam funkčních požadavků}\label{sec:seznamFunkčníchPožadavků}

Před začátkem návrh jakékoli aplikace je důležité definovat funkční požadavky.
Funkční požadavky slouží k vymezení toho, co má aplikace umět a co už umět nemusí.
Jednotlivé funkční požadavky jsou dále rozšířeny uživatelskými případy, které definují použití jednotlivých funkcí aplikace.

% TODO: přidat diagram funkčních pořadavků

\paragraph{F1 -- Správa uživatelů}

Tento požadavek představuje možnost vytvoření uživatelského účtu a jeho používání.
Každý uživatel bude mít uživatelské jméno, které je pro každého uživatele unikátní a slouží jako jeho identifikátor mezi ostatními uživateli.
Uživatel si může své přihlašovací i jiné údaje po přihlášení kdykoliv změnit.

\paragraph{F2 -- Správa dokumentů}

Přihlášený uživatel si může nechat zobrazit vytvořené dokumenty, vytvořený dokument smazat, či vytvořit dokument nový.
Dokumenty jsou vázány na uživatele, který ho vytvořil (dále jen majitel dokumentu), a dokument nelze mezi uživateli přesouvat.

\paragraph{F2 -- Editace dokumentů}

Uživatelé mohou upravovat jednotlivé dokumenty ve skutečném čase spolu s ostatními uživateli.
U dokumentu bude k dispozici komunikační vlákno, kde mohou uživatelé, kteří dokument upravují, komunikovat mezi sebou.
Uživatelé mohou použít veřejný odkaz dokumentu nebo mohou být k dokumentu pozváni jednotlivě.
Jednotlivý uživatelé, krom majitele, mohou mít možnosti editace dokumentu omezeny pomocí nastavení práv dokumentu.
