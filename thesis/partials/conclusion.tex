% !TEX encoding = UTF-8 Unicode
% -*- coding: UTF-8; -*-
% vim: set fenc=utf-8
%\inputencoding{utf8}

\begin{conclusion}

    V rešeršní části práce byly nejprve analyzovány funkční a nefunkční požadavky tohoto nástroje a byl sestaven jeho doménový model.
    Dále byly stručně představeny zadané technologie a existující webové komunikační protokoly včetně jejich využití a nevýhod.
    Také byla představena problematika synchronizace textů ve skutečném čase spolu s nejčastěji používanými synchronizačními algoritmy, operační transformace a diferenciální synchronizace.
    A rešeršní část byla zakončena představením existujících nástrojů pro kolaborativní editaci textů.

    V praktické části práce byl navržen model pro uložení dat a komponenta kolaborativního textového editoru využívající algoritmu operační transformace.
    Dále byl navržen a implementován prototyp aplikace implementující navrženou komponentu kolaborativního textového editoru.
    Uživatelské rozhraní tohoto prototypu bylo úspěšně uživatelsky otestováno a implementace komponenty editoru v rámci prototypu byla otestována výkonnostně.

    Vzniklý prototyp aplikace umožňuje jednoduchou kolaborativní editaci textů, obsahuje jednoduchou správu dokumentů a uživatelských účtů.
    Jednotlivé dokumenty je možné sdílet pomocí jejich veřejných odkazů a to i pouze v režimu náhledu, kdy je dokument také synchronizován ve skutečném čase.

    V budoucnosti by bylo možné navrženou komponentu editoru zbavit závislostí na knihovnách třetích stran (OT.js a Material-UI) a publikovat jako samostatný balíček do npm (JavaScript package manager) registru.
    Takto implementovanou knihovnu by mohli jednoduše využívat i ostatní vývojáři, kteří chtějí vytvořit kolaborativní nástroj s podobnými vlastnostmi jako má implementovaný prototyp aplikace.

    % uvedení cílů či zaměření práce
    % způsob a míra splnění cílů
    % optional: výhled do budoucna
\end{conclusion}
