% !TEX encoding = UTF-8 Unicode
% -*- coding: UTF-8; -*-
% vim: set fenc=utf-8
%\inputencoding{utf8}

\chapter{Cíl práce}

% Cílem rešeršní části práce je
Cílem rešeršní části práce je analýza požadavků, následné seznámení se zadanými technologiemi a rozbor existujících webových komunikačních protokolů.
Dalším cílem je analýza problematiky kolaborativní editace textů a nejčastěji používaných synchronizačních algoritmů.
Ale také rozbor existujících aplikací, které umožňují editaci ve skutečném čase, a analýza doménového modelu.


% Dokumentace a best practise zadaných technologií (HTML5, Javascript, ReactJS, NodeJS)
% Analýza a výběr algoritmů pro kolaborativní spolupráci (OT vs DS)
% Analýza a výběr existujících webových real-time protokolů (http server push - WebSocket, long pooling, pushlet)


% Cílem praktické části je
Cílem praktické části je navržení modelu pro uložení dat a prototypu komponenty kolaborativního textového editoru ve skutečném čase.
Další cílem je implementace a použití navržené komponenty.
A následně uživatelské otestování a vyhodnocení kvality implementovaného řešení.

% Navrhněte model uložení textů, model pro uložení informací o uživatelích a model pro editační změnu.
% Na základě návrhu implementujte prototyp takového nástroje.
% Proveďte uživatelské otestování výsledku a vyhodnoťte kvality a nedostatky vašeho řešení.
