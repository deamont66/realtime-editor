% !TEX encoding = UTF-8 Unicode
% -*- coding: UTF-8; -*-
% vim: set fenc=utf-8
%\inputencoding{utf8}

% Cílem rešeršní části práce je
Cílem rešeršní části práce je seznámení se zadanými technologiemi, které budou využity při implementaci prototypu.
Dalším cílem je analýza problematiky kolaborativní editace textů a rozbor existujících webových real-time protokolů.


% Dokumentace a best practise zadaných technologií (HTML5, Javascript, ReactJS, NodeJS)
% Analýza a výběr algoritmů pro kolaborativní spolupráci (OT vs DS)
% Analýza a výběr existujících webových real-time protokolů (http server push - WebSocket, long pooling, pushlet)

% Cílem praktické části je
Cílem praktické části je navržení modelu pro uložení dat, implementace prototypu a následné uživatelské otestování a vyhodnocení kvality implementovaného řešení.


% Navrhněte model uložení textů, model pro uložení informací o uživatelích a model pro editační změnu.
% Na základě návrhu implementujte prototyp takového nástroje.
% Proveďte uživatelské otestování výsledku a vyhodnoťte kvality a nedostatky vašeho řešení.
