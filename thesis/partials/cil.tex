% !TEX encoding = UTF-8 Unicode
% -*- coding: UTF-8; -*-
% vim: set fenc=utf-8
%\inputencoding{utf8}

\chapter{Cíl práce}\label{ch:cílPráce}

% Cílem rešeršní části práce je
Cílem rešeršní části práce je analýza požadavků a sestavení doménového modelu nástroje pro kolaborativní editaci textů.
Dále seznámení se zadanými technologiemi a rozbor existujících webových komunikačních protokolů.
Cílem je také analýza problematiky synchronizace textů ve skutečném čase a nejčastěji používaných synchronizačních algoritmů.
Ale i rozbor již existujících nástrojů, které umožňují kolaborativní editaci textů ve skutečném čase.


% Dokumentace a best practise zadaných technologií (HTML5, Javascript, ReactJS, NodeJS)
% Analýza a výběr algoritmů pro kolaborativní spolupráci (OT vs DS)
% Analýza a výběr existujících webových real-time protokolů (http server push - WebSocket, long pooling, pushlet)


% Cílem praktické části je
Cílem praktické části práce je navrhnout model pro uložení dat a komponentu kolaborativního textového editoru.
Dále také navrhnout a implementovat prototyp aplikace s použitím navržené komponenty editoru.
Implementovaný prototyp následně uživatelsky otestovat a vyhodnotit jeho kvality a nedostatky.

% Navrhněte model uložení textů, model pro uložení informací o uživatelích a model pro editační změnu.
% Na základě návrhu implementujte prototyp takového nástroje.
% Proveďte uživatelské otestování výsledku a vyhodnoťte kvality a nedostatky vašeho řešení.
