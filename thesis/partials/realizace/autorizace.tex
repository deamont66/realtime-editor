% !TEX encoding = UTF-8 Unicode
% -*- coding: UTF-8; -*-
% vim: set fenc=utf-8
%\inputencoding{utf8}

\section{Autorizace}\label{sec:autorizace}

Protože v navrženém prototypu aplikace nejsou rozlišovány žádné pokročilejší uživatelské role, je naprosto dostačující jednoduchá kontrola přihlášení uživatele při přístupu k jednotlivých koncovým bodů.
Tuto kontrolu provádí \texttt{Middleware} moduly validující každý příchozí požadavek od klienta.
Pokud se uživatel pokouší přistoupit ke koncovému bodu, jenž vyžaduje přihlášeného (resp. nepřihlášeného) uživatele, modul \texttt{AuthMiddleware} zkontroluje zda-li je uživatel přihlášen (resp. odhlášen) a pak teprve serverová část aplikace pokračuje ve zpracování požadavku.

Složitější situace nastává v omezení přístupu uživatelů k jednotlivým dokumentům.
Uživatel může být k editaci dokumentu přizván a jeho práva, tak nemusí odpovídat globálnímu nastavení práv pro sdílení dokumentu (sdílení pomocí veřejného odkazu).
Z tohoto důvodu jsem implementoval statickou třídu \texttt{DocumentVoter} (viz třídní diagram na obrázku~\ref{fig:DocumentVoter}).
Metody této třídy přijímají objekt uživatele (nebo hodnotu \texttt{null}, pokud uživatel není přihlášený) a objekt dokumentu, pro který chceme zjistit uživatelovo oprávnění.
Například metoda \texttt{DocumentVoter.can()} vrací Promise končící úspěchem pouze v případě, kdy má předaný uživatel dostatečné oprávnění k provedení zvolené akce nad předaným dokumentem.

\begin{figure}[ht!]
    \centering
    \includegraphics[width=\textwidth]{partials/realizace/DocumentVoter.pdf}
    \caption{Třídní diagram třídy DocumentVoter}\label{fig:DocumentVoter}
\end{figure}

Složitým problémem autorizace je omezení přístupu k samotnému editoru dokumentů a to nejen v případě, kdy uživatel nemá k dokumentu oprávnění, ale i v případě, kdy dokument vůbec neexistuje.
Klientská část aplikace nemá přístup k DB a nemůže tedy zjistit, zda-li se má editor vůbec aktuálnímu uživateli zobrazit, či jaká část má zůstat skryta, protože k ní uživatel nemá dostatečné oprávnění.
Proto klientská část pro každý dokument zahájí spojení se serverem, jenž po přijetí zprávy Připojení k dokumentu (viz sekce~\ref{sec:komunikaceVeSkutečnémČase} o komunikaci) rozhodne zda-li uživatele připojí (a vrátí seznam operací, ke kterým má uživatel dostatečné oprávnění) nebo odešle zprávu Chyba a spojení ukončí.

Z tohoto důvodu se může stát, že aplikace před zobrazení chyby o nenalezení dokumentu na zlomek vteřiny zobrazí editor dokumentu.
Editor je v uzamčeném stavu a neumožňuje uživateli editaci, ani žádnou jinou operaci nad dokumentem, ale jeho zobrazení je nutné pro správné vytvoření instance textového editoru.
