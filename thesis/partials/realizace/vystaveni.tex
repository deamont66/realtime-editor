% !TEX encoding = UTF-8 Unicode
% -*- coding: UTF-8; -*-
% vim: set fenc=utf-8
%\inputencoding{utf8}

\section{Sestavení aplikace}\label{sec:sestaveníAplikace}

Sestavení aplikace označuje postup kompilace, minimalizace a dalších úprav zdrojů před jejich vystavením do produkčního prostředí.
V implementovaném prototypu aplikace je na sebe klientská a serverová část aplikace vázaná právě po sestavení aplikace a jejím vystavení do produkční prostředí (více v následujících podsekcích).

\subsection{Sestavení klientské části}\label{subsec:sestaveníKlientskéČásti}

Klientská část vyžaduje sestavení před každým spuštěním v uživatelském prohlížeči, protože využívá syntaktických prvků jazyka JavaScript, které nejsou prohlížeči plně podporovány (více o verzích jazyka v sekci~\ref{subsec:javascript}).
Pro sestavení jsem využil již připravené konfigurace pro nástroj \texttt{Webpack}, která přichází s použitím balíčku \texttt{create-react-app} přímo od společnosti Facebook.
Balíček \texttt{create-react-app} obsahuje mimo nastavení pro produkční sestavení aplikace, i nastavení pro vývoj aplikace a to včetně automatického znovu načtení stránky v prohlížeči po provedení změn ve zdrojovém kódu.

Sestavení klientské části lze spustit pomocí příkazu \texttt{yarn build} a sestavená klientská část aplikace je následně umístěně ve složce \texttt{client/build-prod}.
Takto sestavenou klientskou část aplikace je možné jednoduše vystavil, protože se jedná oběžné statické webové stránky s jedním hlavním \gls{HTML} souborem.

\subsection{Serverová část a vystavení klientské části}\label{subsec:serverováČástAVystaveníKlientskéČásti}

Serverová část aplikace je připravena na běh v Node.js prostředí (více o prostředí Node.js v sekci~\ref{subsec:nodejs}), ve kterém byla tato část i vyvíjena.
Postup sestavení proto není u serverové části potřebný.

Jediná změna mezi produkčním a vývojovým vystavení serverové části je nastavení běhového prostředí Node.js pomocí proměnné prostředí \texttt{APP\_ENV}, která musí být nastavena na hodnotu \texttt{production} (resp. \texttt{development} pro vývoj).
Tato proměnná zajišťuje správné nastavení session, cookies a protokolování v závislosti na typu vystavení serverové části.

Serverová část aplikace také umožňuje vystavení předem sestavené klientské části.
Pokud serverová část aplikace zpracovává požadavek, který nemá předponu spojenou s \gls{REST} \gls{API} a ani neodpovídá cestě k žádnému souboru sestavené klientské části, vrátí serverová část obsah hlavního \gls{HTML} souboru sestavené klientské části.
Klientská část aplikace se následně po spuštění v prohlížeči uživatele rozhodne, zda daná adresa existuje, či nikoliv.

\subsubsection{Generování HTML metadat}

Klientská část aplikace generuje a aktualizuje určité \gls{HTML} metadata tak, aby jejich obsah odpovídal zobrazenému obsahu aplikace.
Například po otevření seznamu dokumentů uživatel očekává, že se obsah tagu \texttt{title} změní na odpovídající text, stejně jakoby se jednalo o běžnou webovou stránku.
K dosažení tohoto chování klientské části aplikace jsem použil knihovnu \texttt{react-helmet}, která umožňuje aktualizovat metadata podle pořadí vykreslení jednotlivých React komponent.

Tento přístup je dostatečný, dokud aplikaci používá obyčejný uživatel, který nepostřehne prvotní aktualizaci metadat po spuštění aplikace.
Bohužel ale není dostatečný v případě, kdy k aplikaci přistupují roboti sociálních sítí a webových vyhledávačů, kteří při procházení a indexování internetu nespouštějí JavaScriptový kód.
Pro ně se metadata vůbec nenastaví a při sdílení veřejného odkazu dokumentu na sociálních sítích se objeví pouze jejich původní obsah, který se sdíleným odkazem nemusí souviset.

Chtěl jsem, aby prototyp aplikace při sdílení veřejného odkazu dokumentu zobrazil na sociálních sítích minimálně název tohoto dokumentu.
Toho jsem dosáhl pomocí generování metadat na straně serveru, které jsou následně vloženy do hlavního \gls{HTML} souboru sestavené klientské části a odeslány klientovi.
Robot, který takový odkaz navštíví, získá \gls{HTML} soubor s vygenerovanými metadaty a to bez nutnosti spustit jakýkoliv JavaScript kód klientské části aplikace.

\subsection{Sestavení pomocí Dockeru}\label{subsec:sestaveníPomocíDockeru}

Celý proces sestavení prototypu aplikace jsem zapouzdřil použitím vizualizačního nástroje Docker, který umožňuje definovat aplikace jako takzvané Docker obrazy.
Tyto obrazy následně běží v takzvaných Docker kontejnerech, které jsou navzájem nezávislé a je možné je jednoduše spravovat (spouštět, duplikovat a tak podobně).

Pro prototyp aplikace jsou připravené dva druhy obrazů, vývojový a produkční.
Tyto obrazy obsahují veškeré závislosti klientské i serverové části aplikace.
Připravené konfigurace obrazů a kontejnerů aplikace za pomoci nástroje \texttt{docker-compose} obsahují mimo samotný prototyp aplikace i DB kontejner s mongoDB (více o DB v sekci~\ref{subsec:databáze}).

Vývojový obraz je připraven pro rychlý a jednoduchý vývoj aplikace bez nutnosti instalace jakýchkoliv nástrojů (kromě Dockeru samotného).
Tento obraz lze vytvořit pomocí připraveného příkazu \mintinline{bash}{docker-compose up app_dev}, který vytvoří samotný obraz a jeho kontejner, který následně spustí.
Kontejner a v něm běžící aplikace naslouchá změnám ve zdrojových souborech aplikace a při jejich uložení aplikaci znovu automaticky sestaví.

Produkční obraz je velmi podobný obrazu vývojovému, liší se především v nastavení běhového prostředí a uzavřenosti kontejneru.
Tento obraz lze vytvořit pomocí připraveného příkazu \mintinline{bash}{docker-compose up app_prod}, který vytvoří samotný obraz a jeho kontejner, který následně spustí.
Produkční obraz je připraven pro jeho distribuci mezi dalšími stroji a nese s sebou vše potřebné pro běh aplikace (závislosti, nástroje i zdrojový kód).
Proto je možné prototyp aplikace rychle nasadit bez nutnosti instalace prostření a jeho konfigurace na produkčním stroji.
Pro aktualizaci aplikace ze změněných zdrojových kódů je potřeba produkční obraz znovu vytvořit (není zde žádné automatické znovu sestavení).
