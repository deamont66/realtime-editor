% !TEX encoding = UTF-8 Unicode
% -*- coding: UTF-8; -*-
% vim: set fenc=utf-8
%\inputencoding{utf8}

\section{Autentizace}\label{sec:autentizace}

Využití knihovny passport.js.

\subsection{Přihlášení pomocí hesla}

\subsubsection{Uložení hesla}

Použití hašovací funkce \texttt{bcrypt} a jeho výhody oproti použití \enquote{rychlých} hašovacích funkcí jako jsou funkce standartu sha (dedikovaný hardware a špatně nastavitelná obtížnost výpočtu).

\subsubsection{Odhad obtížnosti hesla}

Knihovna zxcvb od společnost Dropbox.
Problém s vynucování speciálních znaků v heslech (například hesla typu P@ssw0rd, které nemají téměř žádnou přidanou obtížnost díky použití známých substitucí).

\subsubsection{Zapomatování uživatele}

Cookie remember me token a jeho konzumace/generace při každém požadavku (ochrana cookie hijacking).

\subsection{Ostatní možnosti přihlášení}

Napojení a použití s knihovnou passport.js.
Autentizace pomocí ČVUT hesla a sociálních sítí (OAuth2).

\subsubsection{ČVUT heslo}

Portál \hyperref{auth.fit.cvut.cz }(https://auth.fit.cvut.cz/manager) a implementace Provider třídy pro passport.js.

\subsubsection{Sociální sítě}

Použití již vytvořených Provider tříd od tvůrce knihovny passport.js a problémy s nepovinným emailem u autentizace pomocí Facebook účtu.

