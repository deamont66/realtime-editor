% !TEX encoding = UTF-8 Unicode
% -*- coding: UTF-8; -*-
% vim: set fenc=utf-8
%\inputencoding{utf8}

\section{Autentizace}\label{sec:autentizace}

Autentizace označuje proces ověření identity uživatele.
Pro navržený prototyp aplikace se jedná o proces přihlášení uživatele.

K implementaci autentizace jsem využil knihovnu passport.js, která ulehčuje implementaci autentizace pro knihovnu express.js a prostření Node.js (více o prostředí v sekci~\ref{subsec:nodejs}).
Passport.js umožňuje použití autentizačních tříd, které implementují jednotlivé způsoby webové autentizace.
Díky modulárnímu návrhu je možné používat různé způsoby autentizace, bez nutnosti změn zbytku aplikace.

Každá autentizační třída přijímá konfigurační objekt a funkci, která volána pro ověření identity.
Předaná funkce vyhledá uživatele v DB podle identifikátoru, který získá od autentizační třídy, a vrátí objekt uživatele, který má být přihlášen, nebo chybu.

Instanci knihovny passport.js musí být předány funkce pro serializaci a deserializaci uživatele.
V prototypu aplikace jsou uživatelé deserializováni pouze na jejich DB identifikátor a serializováni pomocí jejich opětovného vyhledání v DB.

\subsection{Přihlášení pomocí hesla}\label{subsec:přihlášeníPomocíHesla}

Základním způsobem autentizace v navrženém prototypu je přihlášení pomocí uživatelského jména a hesla.
Knihovna passport.js pro tento způsob poskytuje připravenou třídu \texttt{LocalStrategy}, která požaduje pouze jména parametrů obsahující jméno a heslo přihlašovaného uživatele.

Autorizační funkci jsou předány hodnoty parametrů a uživatel je podle uživatelského jména vyhledán v DB.
Následně dojde k validaci přihlašovacího hesla (viz následující sekce~\ref{subsubsec:uloženíHesla}) a v případě úspěchu je uživatel přihlášen.

\subsubsection{Uložení hesla}\label{subsubsec:uloženíHesla}

Pro možnost validace přihlašovacího hesla je potřeba nějakým způsobem heslo zaznamenat.
Z bezpečnostních důvodů není vhodné v DB uchovávat hesla v čitelné podobě, ani v podobě z které by bylo možné heslo jednoduše získat.

Při implementaci autentizace pomocí hesla jsem se rozhodl pro použití kryptografické hašovací funkce, která poskytuje jednosměrnou transformaci vstupu na téměř náhodný výstup.
Kryptografické hašovací funkce se od běžných hašovacích funkcí nezaměřují na rychlost, ale na kryptografické vlastnosti funkce.

Použil jsem hašovací funkci \texttt{bcrypt}, která se řadí mezi pomalé kryptografické hašovací funkce.
Tato funkce (na rozdíl od rychlých funkcí standartu SHA-2) má nastavitelný počet iterací, což umožňuje navyšovat její obtížnost vypočtu.
Vyšší obtížnost výpočtu je vhodná pro hašování hesel z důvodu zpomalení útoku typu brutal force (opakované a náhodné pokusy o uhodnutí hesla).

\subsubsection{Odhad obtížnosti hesla}

Při změně přihlašovacího hesla jsou dnes běžné minimální požadavky na složitost hesla.
Tyto požadavky jsou však podle~\cite{dropbox:zxcv} však většinou nedostatečné a neodpovídají skutečné době potřebné pro jejich uhodnutí.
Například heslo \texttt{P@ssw0rd} oproti \texttt{Password} nepřináší téměř žádnou přidanou obtížnost, jelikož bylo odvozenou použitím pouze známých a častých substitucí.

Proto jsem se rozhodl do prototypu přidat odhad obtížnosti hesla, který využívá knihovnu zxcv od společnosti Dropbox.
Knihovna umožňuje nastavit minimální časovou hodnotu, za kterou je odhadované prolomení hesla, a podle skóre testovaného hesla rozhodnou, zda-li je dostatečně obtížné.
V prototypu aplikace je tato hodnota nastavena, tak aby bylo zabráněno jen použití opravdu jednoduchých hesel.
Navíc je také odhadovaná doba prolomení zobrazena přímo u pole pro zadání nového hesla, tak aby si mohl uživatel sám zvolit jaká hodnota je pro něj přijatelná.

\subsubsection{Zapamatování uživatele}

Cookie remember me token a jeho konzumace/generace při každém požadavku (ochrana cookie hijacking).

\subsection{Ostatní možnosti přihlášení}

Napojení a použití s knihovnou passport.js.
Autentizace pomocí ČVUT hesla a sociálních sítí (OAuth2).

\subsubsection{ČVUT heslo}

Portál \hyperref{auth.fit.cvut.cz }(https://auth.fit.cvut.cz/manager) a implementace Provider třídy pro passport.js.

\subsubsection{Sociální sítě}

Použití již vytvořených Provider tříd od tvůrce knihovny passport.js a problémy s nepovinným emailem u autentizace pomocí Facebook účtu.

