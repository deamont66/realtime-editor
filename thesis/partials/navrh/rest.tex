% !TEX encoding = UTF-8 Unicode
% -*- coding: UTF-8; -*-
% vim: set fenc=utf-8

\section{REST komunikace}\label{sec:restKomunikace}

Základním způsobem komunikace mezi klientskou a serverovou části aplikace je \gls{REST} komunikace.

K implementaci jednotlivých koncových bodů, ale i obecnému zpracování všech \gls{HTTP} požadavků, jsem se rozhodl použít již připravené řešení v podobě Node.js knihovny, či balíčku knihoven (dále jen framework).
Pro prostření Node.js existuje mnoho frameworků pro implementaci \gls{REST} \gls{API}, podle~\cite{node:framework} jsem výběr zúžil na 3 neoblíbenější Node.js frameworky: Sails.js, Express.js a Hapi.js.

\subsection{Sails.js}\label{subsec:sails.js}

Sails.js je plnohodnotný webový \gls{MVC} framework pro Node.js.
Jeho integrovanou součástí je Waterline \gls{ORM}, který umožňuje použít téměř libovolný \gls{SŘBD}.

Waterline je, ale také záporem celého frameworku, protože není mezi vývojáři příliš rozšířený a občas ho není jednoduché použít (například mapování vnořených objektů).

\subsection{Express.js}\label{subsec:express.js}

Express.js je velice rozšířený, jednoduchý a lehký framework pro Node.js.
V základní konfiguraci obsahuje pouze základní logiku ohledně zpracování \gls{HTTP} požadavků.

Ale jako každý framework má i své nevýhody.
Mezi hlavní nevýhody patří nepříliš propracované zpracování chybových stavů nebo také nedostatečná kódová nezávislost, což může komplikovat další vývoj a znovupoužitelnost částí aplikace.

\subsection{Hapi.js}\label{subsec:hapi.js}

Hapi.js je framework pro Node.js vyvíjený společností WalMart.
Byl vytvořen jako přímá náhrada frameworku Express.js a snaží se řešit jeho nedostatečnou kódovou nezávislost použitím modulární architektury.

Modulární architektura tento problém sice řeší, ale přidává do frameworku vysokou složitost návrhu.
Tato složitost je pro většinu projektů zbytečně vysoká a nevyrovná se přidané hodnotě, která přichází oproti použití Express.js.

\subsection{Výběř frameworku}\label{subsec:výběřFrameworku}

Nakonec jsem se rozhodl použít framework Express.js, převážně kvůli jeho jednoduchosti a rozšířenosti mezi vývojáři, která může být nápomocna hlavně při řešení potenciálního problému.
Jelikož mám navrhnout a implementovat pouze prototyp aplikace, nepovažuji za vhodné zvolit robustní framework typu Sail.js, či Hapi.js.

\subsection{Seznam koncových bodů}\label{subsec:seznamKoncovýchBodů}

\subsubsection{POST /api/auth}

Koncový bod zajišťující registraci nového uživatele.
Tento koncový bod přijímá parametry uživatelské jméno, heslo a email.

\subsubsection{POST /api/auth/signIn}

Koncový bod pro přihlášení pomocí parametru uživatelské jméno a heslo.

\subsubsection{DELETE /api/auth}

Koncový bod pro odhlášení aktuálně přihlášeného uživatele.

\subsubsection{POST /api/auth/forgotPassword}

Koncový bod sloužící k obnově zapomenutého hesla.
Přijímá parametry email a uživatelské jméno.
Pokud je nalezen uživatel se přijatými údaji je mu odeslán email včetně postupu obnovy hesla pomocí unikátního vygenerovaného kódem.

\subsubsection{GET /api/auth/forgotPassword/:token}

Koncový bod zjišťující existenci předaného parametru token.

\subsubsection{PUT /api/auth/forgotPassword/:token}

Koncový bod, který umožňuje za podmínky validního parametru token provést změnu hesla na nové zaslané heslo jako parametr.

\subsubsection{GET /api/user}

Koncový bod pro zjištění informací o aktuálně přihlášeném uživateli.

\subsubsection{PUT /api/user}

Koncový bod umožňující aktualizaci jednotlivých parametrů přihlášeného uživatele.
Přijímanými parametry jsou uživatelské jméno, email, nové heslo a aktuální heslo.
Veškeré změny musí být potvrzeny platným aktuální heslem.

\subsubsection{GET /api/user/document-settings}

Koncový bod pro zjištění výchozího nastavení pro nové dokumenty aktuálně přihlášeného uživatele.

\subsubsection{PUT /api/user/document-settings}

Koncový bod umožňující změnu jednotlivých polí výchozího nastavení pro dokumenty aktuálně přihlášeného uživatele.
Přijímanými parametry jsou všechny pole entity DocumentSettings z databázového schématu na obrázku~\ref{fig:DB_model} (krom pole \_id).

\subsubsection{POST /api/document}

Koncový bod umožňující aktuálně přihlášenému uživateli vytvořit nový dokument.

\subsubsection{GET /api/document}

Koncový bod pro získání dokumentů vytvořených aktuálně přihlášeným uživatelem.

\subsubsection{GET /api/document/last}

Koncový bod pro získání dostupných dokumentů, ke kterým v minulosti přistoupil aktuálně přihlášený uživatel.

\subsubsection{GET /api/document/shared}

Koncový bod pro získání dokumentů, ke kterým byl aktuálně přihlášený uživatel přizván.

\subsubsection{GET /api/document}

Koncový bod pro získání dokumentů vytvořených aktuálně přihlášeným uživatelem (je jejich vlastníkem).

\subsubsection{GET /api/document/:documentId/messages}

Koncový bod pro získání zpráv ohledně dokumentu identifikovaného pomocí parametru documentId.
Pro jeho použití musím mít uživatel dostatečná práva v rámci dokumentu pro přístup ke zprávám.
Počet vrácených zpráv lze ovlivnit parametrem number a čas odeslání poslední vrácené zprávy lze určit parametrem lastDate.

\subsubsection{POST /api/document/:documentId/messages}
Koncový bod umožňující vytvoření nové zprávy pro dokument identifikovaný pomocí parametru documentId s textem určeným parametrem message.
Pro jeho použití musím mít uživatel dostatečná práva v rámci dokumentu pro přístup ke zprávám.

\subsubsection{GET /api/document/:documentId/rights}

Koncový bod pro získání informací ohledně oprávnění a jednotlivých pozvánek dokumentu identifikovaného pomocí parametru documentId.
Pro jeho použití musím mít uživatel dostatečná práva v rámci dokumentu pro přístup k pozvánkám a sdílení.

\subsubsection{PUT /api/document/:documentId/rights}

Koncový bod pro úpravu oprávnění pro uživatele přistupující k dokumentu (identifikovaného pomocí parametru documentId) pomocí veřejného odkazu.
Pro jeho použití musím mít uživatel dostatečná práva v rámci dokumentu pro přístup k pozvánkám a sdílení.

\subsubsection{PUT /api/document/:documentId/rights/invite}

Koncový bod pro úpravu oprávnění pro přizvaného uživatele k dokumentu (identifikovaného pomocí parametru documentId).
Uživatel je identifikován pomocí parametru uživatelské jméno.
Pro jeho použití musím mít uživatel dostatečná práva v rámci dokumentu pro přístup k pozvánkám a sdílení.

\subsubsection{DELETE /api/document/:documentId/rights/:toUserId}

Koncový bod umožňující odstranění pozvánky pro uživatele (identifikovaného pomocí parametru toUserId) k dokumentu (identifikovaného pomocí parametru documentId).
Pro jeho použití musím mít uživatel dostatečná práva v rámci dokumentu pro přístup k pozvánkám a sdílení.

\subsubsection{DELETE /api/document/:documentId}

Koncový bod umožňující trvalé odstranění dokumentu identifikovaného pomocí parametru documentId.
Pro jeho použití musí být aktuálně přihlášený uživatel vlastníkem dokumentu.

\subsubsection{GET /locales/:lang/translation.json}

Koncový bod umožňující stažení textových překladů pro webové rozhraní.
Jazyk vrácených textů je určen pomocí parametru lang (například hodnota cs vrátí českou jazykovou mutaci textů).
