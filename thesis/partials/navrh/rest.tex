% !TEX encoding = UTF-8 Unicode
% -*- coding: UTF-8; -*-
% vim: set fenc=utf-8

\section{REST komunikace}\label{sec:restKomunikace}

Základním způsobem komunikace mezi klientskou a serverovou části aplikace je \gls{REST} komunikace.

K implementaci jednotlivých koncových bodů, ale i obecnému zpracování všech \gls{HTTP} požadavků, jsem se rozhodl použít již připravené řešení v podobě Node.js knihovny, či balíčku knihoven (dále jen framework).
Pro prostření Node.js existuje několik frameworků pro jednoduchou implementaci REST \gls{API}, podle~\cite{node:framework} jsem vybíral ze 3 neoblíbenějších frameworků: Sails.js, Express.js a Hapi.js.

\subsection{Sails.js}\label{subsec:sails.js}

Sails.js je plnohodnotný webový \gls{MVC} framework pro Node.js.
Jeho integrovanou součástí je Waterline \gls{ORM}, který umožňuje použít téměř libovolný \gls{SŘBD}.

Waterline je, ale také záporem celého frameworku, protože není mezi vývojáři příliš rozšířený a občas ho není jednoduché použít (například mapování vnořených objektů).

\subsection{Express.js}\label{subsec:express.js}

Express.js je velice rozšířený, jednoduchý a lehký framework pro Node.js.
V základní konfiguraci obsahuje pouze základní logiku ohledně zpracování \gls{HTTP} požadavků.

Ale jako každý framework má i své nevýhody.
Mezi hlavní nevýhody patří nepříliš propracované zpracování chybových stavů nebo také nedostatečná kódová nezávislost, což může komplikovat další vývoj a znovupoužitelnost částí aplikace.

\subsection{Hapi.js}\label{subsec:hapi.js}

Hapi.js je framework pro Node.js vyvíjený společností WalMart.
Byl vytvořen jako přímá náhrada frameworku Express.js a snaží se řešit jeho nedostatečnou kódovou nezávislost použitím modulární architektury.

Modulární architektura tento problém sice řeší, ale přidává do frameworku vysokou složitost návrhu.
Tato složitost je pro většinu projektů zbytečně vysoká a nevyrovná se přidané hodnotě, která přichází oproti použití Express.js.

\subsection{Výběř frameworku}\label{subsec:výběřFrameworku}

Jelikož mám navrhnout a implementovat pouze prototyp aplikace, nepovažuji za nutné volit robustní framework typu Sail.js, či Hapi.js.
Nakonec jsem se rozhodl použít framework Express.js, převážně kvůli jeho jednoduchosti a rozšířenosti mezi vývojáři, která může být nápomocna hlavně při řešení potenciálního problému.

\subsection{Seznam koncových bodů}\label{subsec:seznamKoncovýchBodů}

\subsubsection{DELETE /api/auth}

