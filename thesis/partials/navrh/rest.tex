% !TEX encoding = UTF-8 Unicode
% -*- coding: UTF-8; -*-
% vim: set fenc=utf-8

\section{REST komunikace}\label{sec:restKomunikace}

Základním způsobem komunikace mezi klientskou a serverovou části aplikace je \gls{REST} komunikace.

K implementaci jednotlivých koncových bodů, ale i obecnému zpracování všech \gls{HTTP} požadavků, jsem se rozhodl použít již připravené řešení v podobě Node.js knihovny, či balíčku knihoven (dále jen framework).
Pro prostření Node.js existuje mnoho frameworků pro implementaci \gls{REST} \gls{API}, podle~\cite{node:framework} jsem výběr zúžil na 3 neoblíbenější Node.js frameworky: Sails.js, Express.js a Hapi.js.

\subsection{Sails.js}\label{subsec:sails.js}

Sails.js je plnohodnotný webový \gls{MVC} framework pro Node.js.
Jeho integrovanou součástí je Waterline \gls{ORM}, který umožňuje použít téměř libovolný \gls{SŘBD}.

Waterline je, ale také záporem celého frameworku, protože není mezi vývojáři příliš rozšířený a občas ho není jednoduché použít (například mapování vnořených objektů).

\subsection{Express.js}\label{subsec:express.js}

Express.js je velice rozšířený, jednoduchý a lehký framework pro Node.js.
V základní konfiguraci obsahuje pouze základní logiku ohledně zpracování \gls{HTTP} požadavků.

Ale jako každý framework má i své nevýhody.
Mezi hlavní nevýhody patří nepříliš propracované zpracování chybových stavů nebo také nedostatečná kódová nezávislost, což může komplikovat další vývoj a znovupoužitelnost částí aplikace.

\subsection{Hapi.js}\label{subsec:hapi.js}

Hapi.js je framework pro Node.js vyvíjený společností WalMart.
Byl vytvořen jako přímá náhrada frameworku Express.js a snaží se řešit jeho nedostatečnou kódovou nezávislost použitím modulární architektury.

Modulární architektura tento problém sice řeší, ale přidává do frameworku vysokou složitost návrhu.
Tato složitost je pro většinu projektů zbytečně vysoká a nevyrovná se přidané hodnotě, která framework přináší oproti použití Express.js.

\subsection{Výběř frameworku}\label{subsec:výběřFrameworku}

Nakonec jsem se rozhodl použít framework Express.js, převážně kvůli jeho jednoduchosti a rozšířenosti mezi vývojáři, která může být nápomocna hlavně při řešení potenciálního problému.
Frameworky Sail.js a Hapi.js se svou velikostí hodí spíše pro větší produkční systémy a nejsou příliš vhodné pro implementaci prototypu této aplikace.

\subsection{Seznam koncových bodů}\label{subsec:seznamKoncovýchBodů}

V této sekci je obsažen kompletní výpis \gls{REST} koncových bodů.
Tyto koncové body přijímají různý počet parametrů, které mohou být jako součást cesty požadavku nebo v těle požadavku ve formátu \gls{JSON}, a vrací \gls{HTTP} stavový kód spolu s nepovinnou odpovědí, která je také ve formátu \gls{JSON}.

\subsubsection{Registrace uživatele}

Koncový bod umožňující registraci nového uživatele.
Tento koncový bod přijímá parametry uživatelské jméno, heslo a email.

Více informací o koncovém bodu lze nalézt v tabulce~\ref{tab:POST/api/auth}.

\begin{table}[ht!]\centering
    \caption{Koncový bod Registrace uživatele}\label{tab:POST/api/auth}

    \begin{tabular}{r|p{2.5cm} p{5.5cm}}
        \acrshort{URL}: & \multicolumn{2}{ l }{/api/auth}\\ \hline
        \acrshort{HTTP} metoda: & \multicolumn{2}{ l }{POST}\\ \hline
        \acrshort{URL} parametry: & \multicolumn{2}{ l }{žádné}\\ \hline
        Data parametry: & username & [String]\\
        & email & [String]\\
        & password & [String]\\ \hline
        Úspěch: & 200 & Informace o vytvořeném uživateli\\ \hline
        Neúspěch: & 422 & Popis chyby\\ \hline
    \end{tabular}
\end{table}

\subsubsection{Přihlášení uživatele}

Koncový bod pro přihlášení již registrovaného uživatele pomocí uživatelské jména a hesla.

Více informací o koncovém bodu lze nalézt v tabulce~\ref{tab:POST/api/auth/signIn}.

\begin{table}[ht!]\centering
\caption{Koncový bod Přihlášení uživatele}\label{tab:POST/api/auth/signIn}

\begin{tabular}{r|p{2.5cm} p{5.5cm}}
    \acrshort{URL}: & \multicolumn{2}{ l }{/api/auth/signIn}\\ \hline
    \acrshort{HTTP} metoda: & \multicolumn{2}{ l }{POST}\\ \hline
    \acrshort{URL} parametry: & \multicolumn{2}{ l }{žádné}\\ \hline
    Data parametry: & username & [String]\\
    & password & [String]\\ \hline
    Úspěch: & 200 & Informace o přihlášeném uživateli\\ \hline
    Neúspěch: & 422 & Popis chyby\\ \hline
\end{tabular}
\end{table}

\subsubsection{Ohlášení uživatele}

Koncový bod pro odhlášení aktuálně přihlášeného uživatele.

Více informací o koncovém bodu lze nalézt v tabulce~\ref{tab:DELETE/api/auth}.

\begin{table}[ht!]\centering
\caption{Koncový bod Ohlášení uživatele}\label{tab:DELETE/api/auth}

\begin{tabular}{r|p{2.5cm} p{5.5cm}}
    \acrshort{URL}: & \multicolumn{2}{ l }{/api/auth}\\ \hline
    \acrshort{HTTP} metoda: & \multicolumn{2}{ l }{DELETE}\\ \hline
    \acrshort{URL} parametry: & \multicolumn{2}{ l }{žádné}\\ \hline
    Data parametry: & \multicolumn{2}{ l }{žádné}\\ \hline
    Úspěch: & 204 & \\ \hline
    Neúspěch: & 401, 422 & Popis chyby\\ \hline
\end{tabular}
\end{table}

\subsubsection{Odeslání požadavku na obnovu hesla}

Koncový bod sloužící k obnově zapomenutého hesla.
Přijímá parametry email a uživatelské jméno.

Pokud je nalezen uživatel se přijatými údaji je mu odeslán email obsahující instrukce pro obnovu hesla pomocí unikátního vygenerovaného kódu.

\begin{table}[ht!]\centering
\caption{Koncový bod Odeslání požadavku na obnovu hesla}\label{tab:POST/api/auth/forgotPassword}

\begin{tabular}{r|p{2.5cm} p{5.5cm}}
    \acrshort{URL}: & \multicolumn{2}{ l }{/api/auth/forgotPassword}\\ \hline
    \acrshort{HTTP} metoda: & \multicolumn{2}{ l }{POST}\\ \hline
    \acrshort{URL} parametry: & \multicolumn{2}{ l }{žádné}\\ \hline
    Data parametry: & username & [String]\\
    & email & [String]\\ \hline
    Úspěch: & 200 & Neutrální zpráva o provedení operace \\ \hline
    Neúspěch: & \multicolumn{2}{ l }{Nenastává}\\ \hline
\end{tabular}
\end{table}

\subsubsection{Zjištění platnosti kódu pro obnovu hesla}

Koncový bod zjišťující existenci kódu pro obnovy hesla předaného pomocí parametru token.

\begin{table}[ht!]\centering
\caption{Koncový bod Zjištění platnosti kódu pro obnovu hesla}\label{tab:GET/api/auth/forgotPassword/:token}

\begin{tabular}{r|p{2.5cm} p{5.5cm}}
    \acrshort{URL}: & \multicolumn{2}{ l }{/api/auth/forgotPassword/:token}\\ \hline
    \acrshort{HTTP} metoda: & \multicolumn{2}{ l }{GET}\\ \hline
    \acrshort{URL} parametry: & token & [String]\\ \hline
    Data parametry: & \multicolumn{2}{ l }{žádné}\\ \hline
    Úspěch: & 204 & \\ \hline
    Neúspěch: & 404 & Popis chyby\\ \hline
\end{tabular}
\end{table}

\subsubsection{Obnova hesla pomocí kódu}

Koncový bod, který umožňuje za podmínky validního kódu pro obnovu hesla (parametr token) provést změnu hesla na nové heslo.

\begin{table}[ht!]\centering
\caption{Koncový bod Obnova hesla pomocí kódu}\label{tab:PUT/api/auth/forgotPassword/:token}

\begin{tabular}{r|p{2.5cm} p{5.5cm}}
    \acrshort{URL}: & \multicolumn{2}{ l }{/api/auth/forgotPassword/:token}\\ \hline
    \acrshort{HTTP} metoda: & \multicolumn{2}{ l }{PUT}\\ \hline
    \acrshort{URL} parametry: & token & [String]\\ \hline
    Data parametry: &  newPassword & [String] \\ \hline
    Úspěch: & 204 & \\ \hline
    Neúspěch: & 404, 422 & Popis chyby\\ \hline
\end{tabular}
\end{table}

\subsubsection{Informace o přihlášeném uživateli}\label{subsubsec:GET/api/user}

Koncový bod pro zjištění informací o aktuálně přihlášeném uživateli.

\begin{table}[ht!]\centering
\caption{Koncový bod Informace o přihlášeném uživateli}\label{tab:GET/api/user}

\begin{tabular}{r|p{2.5cm} p{5.5cm}}
    \acrshort{URL}: & \multicolumn{2}{ l }{/api/user}\\ \hline
    \acrshort{HTTP} metoda: & \multicolumn{2}{ l }{GET}\\ \hline
    \acrshort{URL} parametry: & \multicolumn{2}{ l }{žádné}\\ \hline
    Data parametry: & \multicolumn{2}{ l }{žádné}\\ \hline
    Úspěch: & 200 & Informace o přihlášeném uživateli\\ \hline
    Neúspěch: & 401 & Popis chyby\\ \hline
\end{tabular}
\end{table}

\subsubsection{Změna údajů přihlášeného uživatele}

Koncový bod umožňující aktualizaci jednotlivých parametrů přihlášeného uživatele.

Parametry jsou volitelné, stačí tedy odeslat pouze údaje, které mají být změněné.
Veškeré změny musí být potvrzeny platným aktuální heslem (parametr password je tedy povinný vždy).

\begin{table}[ht!]\centering
\caption{Koncový bod Změna údajů přihlášeného uživatele}\label{tab:PUT/api/user}

\begin{tabular}{r|p{2.5cm} p{5.5cm}}
    \acrshort{URL}: & \multicolumn{2}{ l }{/api/user}\\ \hline
    \acrshort{HTTP} metoda: & \multicolumn{2}{ l }{PUT}\\ \hline
    \acrshort{URL} parametry: & \multicolumn{2}{ l }{žádné}\\ \hline
    Data parametry: & username & [String]\\
    & email & [String]\\
    & newPassword & [String]\\
    & password & [String]\\ \hline
    Úspěch: & 204 & \\ \hline
    Neúspěch: & 401, 422 & Popis chyby\\ \hline
\end{tabular}
\end{table}

\subsubsection{Výchozí nastavení dokumentů}

Koncový bod pro zjištění výchozího nastavení pro nové dokumenty aktuálně přihlášeného uživatele.

\begin{table}[ht!]\centering
\caption{Koncový bod Výchozí nastavení dokumentů}\label{tab:GET/api/user/document-settings}

\begin{tabular}{r|p{2.5cm} p{5.5cm}}
    \acrshort{URL}: & \multicolumn{2}{ l }{/api/user/document-settings}\\ \hline
    \acrshort{HTTP} metoda: & \multicolumn{2}{ l }{GET}\\ \hline
    \acrshort{URL} parametry: & \multicolumn{2}{ l }{žádné}\\ \hline
    Data parametry: & \multicolumn{2}{ l }{žádné}\\ \hline
    Úspěch: & 200 & Výchozí nastavení dokumentů\\ \hline
    Neúspěch: & 401 & Popis chyby\\ \hline
\end{tabular}
\end{table}

\subsubsection{Změna výchozího nastavení dokumentů}

Koncový bod umožňující změnu jednotlivých polí výchozího nastavení pro dokumenty aktuálně přihlášeného uživatele.
Přijímanými parametry jsou všechny pole entity DocumentSettings z databázového schématu na obrázku~\ref{fig:DB_model} (kromě pole \_id).

\begin{table}[ht!]\centering
\caption{Koncový bod Změna výchozího nastavení dokumentů}\label{tab:PUT/api/user/document-settings}

\begin{tabular}{r|p{2.5cm} p{5.5cm}}
    \acrshort{URL}: & \multicolumn{2}{ l }{/api/user/document-settings}\\ \hline
    \acrshort{HTTP} metoda: & \multicolumn{2}{ l }{PUT}\\ \hline
    \acrshort{URL} parametry: & \multicolumn{2}{ l }{žádné}\\ \hline
    Data parametry: & theme & [String]\\
    & mode & [Number]\\
    & tabSize & [Number]\\
    & indentUnit & [Number]\\
    & indentWithTabs & [Boolean]\\
    & fontSize & [Number]\\
    & keyMap & [String]\\
    & styleActiveLine & [String]\\
    & lineWrapping & [Boolean]\\
    & lineNumbers & [Boolean]\\ \hline
    Úspěch: & 200 & Výchozí nastavení dokumentů\\ \hline
    Neúspěch: & 401, 422 & Popis chyby\\ \hline
\end{tabular}
\end{table}

\subsubsection{Vytvoření dokumentu}

Koncový bod umožňující aktuálně přihlášenému uživateli vytvořit nový dokument.

\begin{table}[ht!]\centering
\caption{Koncový bod Vytvoření dokumentu}\label{tab:POST/api/document}

\begin{tabular}{r|p{2.5cm} p{5.5cm}}
    \acrshort{URL}: & \multicolumn{2}{ l }{/api/document}\\ \hline
    \acrshort{HTTP} metoda: & \multicolumn{2}{ l }{POST}\\ \hline
    \acrshort{URL} parametry: & \multicolumn{2}{ l }{žádné}\\ \hline
    Data parametry: & \multicolumn{2}{ l }{žádné}\\ \hline
    Úspěch: & 200 & Identifikátor vytvořeného dokumentu\\ \hline
    Neúspěch: & 401 & Popis chyby\\ \hline
\end{tabular}
\end{table}

\subsubsection{Vytvořené dokumenty}

Koncový bod pro získání dokumentů vytvořených aktuálně přihlášeným uživatelem (je jejich vlastníkem).

\begin{table}[ht!]\centering
\caption{Koncový bod Vytvořené dokumenty}\label{tab:GET/api/document}

\begin{tabular}{r|p{2.5cm} p{5.5cm}}
    \acrshort{URL}: & \multicolumn{2}{ l }{/api/document}\\ \hline
    \acrshort{HTTP} metoda: & \multicolumn{2}{ l }{GET}\\ \hline
    \acrshort{URL} parametry: & \multicolumn{2}{ l }{žádné}\\ \hline
    Data parametry: & \multicolumn{2}{ l }{žádné}\\ \hline
    Úspěch: & 200 & Seznam dokumentů\\ \hline
    Neúspěch: & 401 & Popis chyby\\ \hline
\end{tabular}
\end{table}

\subsubsection{Poslední dokumenty}

Koncový bod pro získání dostupných dokumentů, ke kterým v minulosti přistoupil aktuálně přihlášený uživatel.

\begin{table}[ht!]\centering
\caption{Koncový bod Poslední dokumenty}\label{tab:GET/api/document/last}

\begin{tabular}{r|p{2.5cm} p{5.5cm}}
    \acrshort{URL}: & \multicolumn{2}{ l }{/api/document/last}\\ \hline
    \acrshort{HTTP} metoda: & \multicolumn{2}{ l }{GET}\\ \hline
    \acrshort{URL} parametry: & \multicolumn{2}{ l }{žádné}\\ \hline
    Data parametry: & \multicolumn{2}{ l }{žádné}\\ \hline
    Úspěch: & 200 & Seznam dokumentů\\ \hline
    Neúspěch: & 401 & Popis chyby\\ \hline
\end{tabular}
\end{table}

\subsubsection{Sdílené dokumenty}

Koncový bod pro získání dokumentů, ke kterým byl aktuálně přihlášený uživatel přizván.

\begin{table}[ht!]\centering
\caption{Koncový bod Sdílené dokumenty}\label{tab:GET/api/document/shared}

\begin{tabular}{r|p{2.5cm} p{5.5cm}}
    \acrshort{URL}: & \multicolumn{2}{ l }{/api/document/shared}\\ \hline
    \acrshort{HTTP} metoda: & \multicolumn{2}{ l }{GET}\\ \hline
    \acrshort{URL} parametry: & \multicolumn{2}{ l }{žádné}\\ \hline
    Data parametry: & \multicolumn{2}{ l }{žádné}\\ \hline
    Úspěch: & 200 & Seznam dokumentů\\ \hline
    Neúspěch: & 401 & Popis chyby\\ \hline
\end{tabular}
\end{table}

\subsubsection{Komunikační vlákno dokumentu}

Koncový bod pro získání zpráv komunikačního vlákna dokumentu identifikovaného pomocí parametru documentId.
Pro jeho použití musím mít uživatel dostatečná práva v rámci dokumentu pro přístup ke zprávám.

Počet vrácených zpráv lze ovlivnit volitelným \gls{URL} parametrem number a čas odeslání poslední vrácené zprávy lze určit volitelným \gls{URL} parametrem lastDate.

\begin{table}[ht!]\centering
\caption{Koncový bod Komunikační vlákno dokumentu}\label{tab:GET/api/document/:documentId/messages}

\begin{tabular}{r|p{2.5cm} p{5.5cm}}
    \acrshort{URL}: & \multicolumn{2}{ l }{/api/document/:documentId/messages}\\ \hline
    \acrshort{HTTP} metoda: & \multicolumn{2}{ l }{GET}\\ \hline
    \acrshort{URL} parametry: & documentId & [String]\\
    & lastDate & [Date]\\
    & number & [Number]\\ \hline
    Data parametry: & \multicolumn{2}{ l }{žádné}\\ \hline
    Úspěch: & 200 & Seznam zpráv\\ \hline
    Neúspěch: & 403, 404, 422 & Popis chyby\\ \hline
\end{tabular}
\end{table}

\subsubsection{Odeslání nové zprávy}
Koncový bod umožňující vytvoření nové zprávy pro dokument identifikovaný pomocí parametru documentId s textem určeným parametrem message.

Pro jeho použití musím mít uživatel dostatečná práva v rámci dokumentu pro přístup ke zprávám.

\begin{table}[ht!]\centering
\caption{Koncový bod Odeslání nové zprávy}\label{tab:POST/api/document/:documentId/messages}

\begin{tabular}{r|p{2.5cm} p{5.5cm}}
    \acrshort{URL}: & \multicolumn{2}{ l }{/api/document/:documentId/messages}\\ \hline
    \acrshort{HTTP} metoda: & \multicolumn{2}{ l }{POST}\\ \hline
    \acrshort{URL} parametry: & documentId & [String]\\ \hline
    Data parametry: & message & [String]\\ \hline
    Úspěch: & 204 &\\ \hline
    Neúspěch: & 401, 403, 404, 422 & Popis chyby\\ \hline
\end{tabular}
\end{table}

\subsubsection{Práva dokumentu}

Koncový bod pro získání informací ohledně oprávnění a jednotlivých pozvánek dokumentu identifikovaného pomocí parametru documentId.

Pro jeho použití musím mít uživatel dostatečná práva v rámci dokumentu pro přístup k pozvánkám a sdílení.

\begin{table}[ht!]\centering
\caption{Koncový bod Práva dokumentu}\label{tab:GET/api/document/:documentId/rights}

\begin{tabular}{r|p{2.5cm} p{4.5cm}}
    \acrshort{URL}: & \multicolumn{2}{ l }{/api/document/:documentId/rights}\\ \hline
    \acrshort{HTTP} metoda: & \multicolumn{2}{ l }{GET}\\ \hline
    \acrshort{URL} parametry: & documentId & [String]\\ \hline
    Data parametry: & \multicolumn{2}{ l }{žádné}\\ \hline
    Úspěch: & 200 & Informace o sdílení dokumentu\\ \hline
    Neúspěch: & 403, 404 & Popis chyby\\ \hline
\end{tabular}
\end{table}

\subsubsection{Změna práv veřejného odkazu dokumentu}

Koncový bod pro úpravu oprávnění pro uživatele přistupující k dokumentu (identifikovaného pomocí parametru documentId) pomocí veřejného odkazu.

Pro jeho použití musím mít uživatel dostatečná práva v rámci dokumentu pro přístup k pozvánkám a sdílení.

\begin{table}[ht!]\centering
\caption{Koncový bod Změna práv veřejného odkazu dokumentu}\label{tab:PUT/api/document/:documentId/rights}

\begin{tabular}{r|p{2.5cm} p{5.5cm}}
    \acrshort{URL}: & \multicolumn{2}{ l }{/api/document/:documentId/rights}\\ \hline
    \acrshort{HTTP} metoda: & \multicolumn{2}{ l }{PUT}\\ \hline
    \acrshort{URL} parametry: & documentId & [String]\\ \hline
    Data parametry: & shareLinkRights & [Number]\\ \hline
    Úspěch: & 204 & \\ \hline
    Neúspěch: & 403, 404, 422 & Popis chyby\\ \hline
\end{tabular}
\end{table}

\subsubsection{Pozvání uživatele k dokumentu}

Koncový bod pro úpravu oprávnění pro přizvaného uživatele k dokumentu (identifikovaného pomocí parametru documentId).
Pozvaný uživatel je identifikován pomocí uživatelského jména v parametru.

Pro jeho použití musí mít uživatel dostatečná práva v rámci dokumentu pro přístup k pozvánkám a sdílení.

\begin{table}[ht!]\centering
\caption{Koncový bod Pozvání uživatele k dokumentu}\label{tab:PUT/api/document/:documentId/rights/invite}

\begin{tabular}{r|p{2.5cm} p{5.5cm}}
    \acrshort{URL}: & \multicolumn{2}{ l }{/api/document/:documentId/rights/invite}\\ \hline
    \acrshort{HTTP} metoda: & \multicolumn{2}{ l }{PUT}\\ \hline
    \acrshort{URL} parametry: & documentId & [String]\\ \hline
    Data parametry: & rights & [Number]\\
    & to & [String]\\ \hline
    Úspěch: & 204 & \\ \hline
    Neúspěch: & 403, 404, 422 & Popis chyby\\ \hline
\end{tabular}
\end{table}

\subsubsection{Odstranění pozvánky k dokumentu}

Koncový bod umožňující odstranění pozvánky pro uživatele (identifikovaného pomocí parametru toUserId) k dokumentu (identifikovaného pomocí parametru documentId).

Pro jeho použití musím mít uživatel dostatečná práva v rámci dokumentu pro přístup k pozvánkám a sdílení.

\begin{table}[ht!]\centering
\caption{Koncový bod Odstranění pozvánky k dokumentu}\label{tab:DELETE/api/document/:documentId/rights/:toUserId}

\begin{tabular}{r|p{2.5cm} p{5.5cm}}
    \acrshort{URL}: & \multicolumn{2}{ l }{/api/document/:documentId/rights/:toUserId}\\ \hline
    \acrshort{HTTP} metoda: & \multicolumn{2}{ l }{DELETE}\\ \hline
    \acrshort{URL} parametry: & documentId & [String]\\
    & toUserId & [String]\\ \hline
    Data parametry: & \multicolumn{2}{ l }{žádné} \\ \hline
    Úspěch: & 204 & \\ \hline
    Neúspěch: & 403, 404, 422 & Popis chyby\\ \hline
\end{tabular}
\end{table}

\subsubsection{Odstranění dokumentu}

Koncový bod umožňující trvalé odstranění dokumentu identifikovaného pomocí parametru documentId.

Pro jeho použití musí být aktuálně přihlášený uživatel vlastníkem dokumentu.

\begin{table}[ht!]\centering
\caption{Koncový bod Odstranění dokumentu}\label{tab:DELETE/api/document/:documentId}

\begin{tabular}{r|p{2.5cm} p{5.5cm}}
    \acrshort{URL}: & \multicolumn{2}{ l }{/api/document/:documentId}\\ \hline
    \acrshort{HTTP} metoda: & \multicolumn{2}{ l }{DELETE}\\ \hline
    \acrshort{URL} parametry: & documentId & [String]\\ \hline
    Data parametry: & \multicolumn{2}{ l }{žádné} \\ \hline
    Úspěch: & 204 & \\ \hline
    Neúspěch: & 403, 404 & Popis chyby\\ \hline
\end{tabular}
\end{table}

\subsubsection{Překlady webového rozhraní}\label{subsubsec:prekladyWebovehoRozhrani}

Koncový bod umožňující stažení textových překladů pro webové rozhraní.

Jazyk vrácených textů je určen pomocí parametru lang (například hodnota cs vrátí českou jazykovou mutaci textů).

\begin{table}[ht!]\centering
\caption{Koncový bod Překlady webového rozhraní}\label{tab:GET/locales/:lang/translation.json}

\begin{tabular}{r|p{2.5cm} p{5.5cm}}
    \acrshort{URL}: & \multicolumn{2}{ l }{/locales/:lang/translation.json}\\ \hline
    \acrshort{HTTP} metoda: & \multicolumn{2}{ l }{GET}\\ \hline
    \acrshort{URL} parametry: & lang & [String]\\ \hline
    Data parametry: & \multicolumn{2}{ l }{žádné} \\ \hline
    Úspěch: & 200 & \gls{JSON} s textovými překlady \\ \hline
    Neúspěch: & 404 & Popis chyby\\ \hline
\end{tabular}
\end{table}
