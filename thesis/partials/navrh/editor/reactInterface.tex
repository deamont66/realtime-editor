% !TEX encoding = UTF-8 Unicode
% -*- coding: UTF-8; -*-
% vim: set fenc=utf-8

\subsubsection{React.js rozhraní komponenty}

Klientská část komponenty v prototypu je zapouzdřena do React.js komponenty (více o React.js v sekci~\ref{subsec:reactjs}), kterou jsem pojmenoval \texttt{RealtimeEditor}.
React.js komponenta umožňuje klientskou část znovu použití kdekoliv v aplikaci a pomocí modulárního návrhu lze upravit i její vzhled.

React.js komponenta po svém prvním vykreslení vytvoří spojení k serverové části a vytvoří instanci textového editoru CodeMirror, které předá nově vytvořeným instancím tříd \texttt{SocketIOServerAdapter} a \texttt{CodeMirrorEditorAdapter}.
Komponenta dálé vytvoří instanci třídy \texttt{EditorClient}, které předá vytvořené instance tříd \texttt{SocketIOServerAdapter} a \texttt{CodeMirrorEditorAdapter}.
Komponenta ve svém stavu drží infromace o připojených klientech, nastavení a další informace ohledně aktuálně připojeného dokumentu.

\paragraph{Vlastnosti} (anglicky properties) jsou standardním způsobem, jak jednotlivým React komponentám předat informace.
Jedná o analogické řešení jakým jsou atributy u \gls{HTML} entit (kde atributy jsou vlastnosti a entity jsou samotné komponenty).

React.js komponenta \texttt{RealtimeEditor} přijímá jak povinné, tak i nepovinné vlastnosti (viz tabulka~\ref{tab:reactKomponenta}).

\begin{table}[ht!]\centering
\caption{Vlastnosti přijímané React.js komponentou RealtimeEditor}\label{tab:reactKomponenta}

\begin{tabular}{l c c p{5.5cm}}
    Jméno: & Povinná: & Datový typ: & Popis:\\ \hline
    documentId & ano & String & Identifikátor dokumentu\\ \hline
    user & ne & Object & Informace o přihlášeném uživateli\\ \hline
    headerSlot & ne & Node & Komponenta, která se vykreslí na místě hlavičky dokumentu \\ \hline
    menuSlot & ne & Node & Komponenta, která se vykreslí na místě menu dokumentu \\ \hline
    errorSlot & ne & Node & Komponenta, která se vykreslí pokud dojde k chybě \\
\end{tabular}
\end{table}

Vlastnost \texttt{documentId} označuje identifikátor dokumentu, ke kterému se komponenta má připojit.
Vlastnost \texttt{user} je objekt s informacemi o přihlášeném uživateli (očekávaná struktura je stejná jako struktura, kterou vrací \gls{REST} koncový bod~\ref{subsubsec:GET/api/user}).
Pokud není uživatel přihlášen není nutné vlastnost předávat, či stačí předat hodnotu, která se vyhodnotí jako nepravdivá (\texttt{false}, \texttt{null} a další).
Poslední vlastnosti \texttt{headerSlot}, \texttt{menuSlot} a \texttt{errorSlot} označují modulární React.js komponenty, které jsem vykresleny na specifickém místě \texttt{RealtimeEditor} komponenty.
Modulárním komponentám jsou předány vnitřní informace o stavu editoru pomocí jejich vlastností.

Nejjednodušší použití komponenty je ukázané ve výpisu kódu~\ref{lst:RealtimeEditor}.
Tento příklad vykreslí pouze samotný textový editor, který je ovšem aktualizovaný v reálném čase ve všemi uživateli, kteří jsou ve stejnou chvíli připojeni k dokumentu s identifikátorem \enquote{helloWorld}.

\begin{listing}[ht]
    \begin{minted}{javascript}
        export default () => (
            <RealtimeEditor documentId="helloWorld"/>
        )
    \end{minted}

    \caption{Příklad použití komponenty RealtimeEditor}\label{lst:RealtimeEditor}
\end{listing}


