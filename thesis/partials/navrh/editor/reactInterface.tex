% !TEX encoding = UTF-8 Unicode
% -*- coding: UTF-8; -*-
% vim: set fenc=utf-8

\subsubsection{React rozhraní komponenty}

Klientská část komponenty v prototypu je zapouzdřena do React komponenty (více o knihovně React v sekci~\ref{subsec:reactjs}) pojmenované \texttt{RealtimeEditor}.
React komponenta umožňuje klientskou část znovu použít kdekoliv v aplikaci a pomocí modulárního návrhu lze upravit i její vzhled.

React komponenta po svém prvním vykreslení vytvoří spojení k serverové části a vytvoří instanci textového editoru CodeMirror.
Ty následně předá nově vytvořeným instancím tříd \texttt{SocketIOServerAdapter} a \texttt{CodeMirrorEditorAdapter}.
Komponenta dále vytvoří instanci třídy \texttt{EditorClient}, které předá vytvořené instance tříd \texttt{SocketIOServerAdapter} a \texttt{CodeMirrorEditorAdapter}.
Komponenta ve svém stavu drží informace o připojených klientech, nastavení a další informace ohledně aktuálně připojeného dokumentu.

React komponenta \texttt{RealtimeEditor} přijímá jak povinné, tak i nepovinné vlastnosti (viz tabulka~\ref{tab:reactKomponenta}).

\begin{table}[ht!]\centering
\caption{Vlastnosti přijímané React komponentou RealtimeEditor}\label{tab:reactKomponenta}

\begin{tabular}{l c c p{5.5cm}}
    Jméno: & Povinná: & Datový typ: & Popis:\\ \hline
    documentId & ano & String & Identifikátor dokumentu\\ \hline
    user & ne & Object & Informace o přihlášeném uživateli\\ \hline
    headerSlot & ne & Node & Komponenta vykreslená na místě hlavičky dokumentu \\ \hline
    menuSlot & ne & Node & Komponenta vykreslená na místě menu dokumentu \\ \hline
    errorSlot & ne & Node & Komponenta vykreslená pokud dojde k chybě \\
\end{tabular}
\end{table}

Vlastnost \texttt{documentId} označuje identifikátor dokumentu, ke kterému se komponenta má připojit.
Vlastnost \texttt{user} je objekt s informacemi o přihlášeném uživateli (očekávaná struktura je stejná jako struktura získaná pomocí \gls{REST} koncového bodu~\ref{subsubsec:GET/api/user}).
V případě nepřihlášeného uživatele není nutné vlastnost \texttt{user} vůbec předávat, či lze předat libovolnou hodnotu, která se vyhodnotí jako nepravda.
Poslední vlastnosti \texttt{headerSlot}, \texttt{menuSlot} a \texttt{errorSlot} označují modulární React komponenty vykreslené na specifickém místě \texttt{RealtimeEditor} komponenty.
Modulárním komponentám jsou předány vnitřní informace o stavu editoru pomocí jejich vlastností a jsou tak upozorněny i v případě jejich změny.

Nejjednodušší použití komponenty je ukázané ve výpisu kódu~\ref{lst:RealtimeEditor}.
Tento příklad vykreslí pouze samotný textový editor, který je ovšem aktualizovaný v reálném čase se všemi uživateli, kteří jsou ve stejnou chvíli připojeni k dokumentu s identifikátorem \enquote{helloWorld}.

\begin{listing}[ht!]
    \begin{minted}{javascript}
        export default () => (
            <RealtimeEditor documentId="helloWorld"/>
        )
    \end{minted}

    \caption{Příklad použití komponenty RealtimeEditor}\label{lst:RealtimeEditor}
\end{listing}
