% !TEX encoding = UTF-8 Unicode
% -*- coding: UTF-8; -*-
% vim: set fenc=utf-8

\section{Databázové schéma}\label{sec:databázovéSchéma}

Jelikož jsem se rozhodl použít \gls{NoSQL} databázi, nemám jak pevně definovat databázové schéma jako v případě použití relační databáze.
Konzistenci a validitu ukládaných dat musí zajistit sama aplikace a to jak při čtení, tak i při zápisu dat.

Z tohoto důvodu jsem se rozhodl použít \gls{ODM} nástroj Mongoose, který umožňuje v rámci aplikace definovat struktury jednotlivých dokumentů pro MongoDB a jejich validitu kontroluje před každým uložením.
Schéma na obrázku~\ref{fig:DB_model} není databázovým schématem v pravém slova smyslu, ale jedná se o schéma definované za pomoci právě \gls{ODM} Mongoose.

\begin{figure}[ht!]
    \centering
    \includegraphics[width=.95\textwidth]{partials/navrh/DB_model.pdf}
    \caption{Databázové schéma v rámci ODM Mongoose}\label{fig:DB_model}
\end{figure}

Jednotlivé entity schématu přímo reflektují entity z doménového modelu uvedeného v sekci~\ref{sec:domenovyModel}.
Přibyly pouze implementační detaily a atributy jednotlivých relací mezi entitami.
Také se změnily názvy entit a jejich atributů z českého do anglického jazyka tak, aby lepé reflektovaly samotnou implementaci aplikace.

Jedinou podstatnou změnou oproti doménovému modelu, jsou atributy entity \texttt{User}.
Přibyly zde atributy sloužící pro autentizaci pomocí služeb třetích stran (například sociální sítě, či \acrshort{ČVUT} heslo).
Tyto atributy nesou identifikační údaj pro uživatele v rámci dané autentizační služby třetí strany, jako je například facebookId pro sociální síť Facebook, či \acrshort{ČVUT} uživatelské jméno pro přihlášení pomocí \acrshort{ČVUT} hesla.

% TODO: upravit cestu podle finální přílohy
Kompletní definici schématu lze nalézt spolu s ostatními zdrojovými kódy na přiloženém CD ve složce \texttt{/realtime-editor/app/src/model}.
Na definici lze i mimo jiné pozorovat validační pravidla pro jednotlivé atributy, jako jsou například jejich číselné rozsahy, maximální délky a tak dále.
