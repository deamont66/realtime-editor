% !TEX encoding = UTF-8 Unicode
% -*- coding: UTF-8; -*-
% vim: set fenc=utf-8

\section{Komunikace ve skutečném čase}\label{sec:komunikaceVeSkutečnémČase}

K implementaci komunikace ve skutečném čase existuje více způsoby, ale každý má své kompromisy (více o push technologiích v sekci~\ref{subsec:pushTechnologie}).
Pro komunikaci jsem chtěl použít technologii WebSocket (viz sekce~\ref{subsubsec:websocket}), kvůli podpoře full duplexní oboustranné komunikace, ale zároveň jsem chtěl umožnit použít aplikace uživatelům se starším webovým prohlížečem, či mobilním zařízením.

Z tohoto důvodu jsem se rozhodl pro použití knihovny Socket.io, která je je postavena na transportní knihovně Engine.io.
Knihovna Socket.io poskytuje ucelené \gls{API} pro použití technologie WebSocket, pro všechny modelní prohlížeče, ale i pro prohlížeče, které technologii WebSocket dosud nepodporují a to díky transparentnímu použití záložní komunikace pomocí long pollingu (viz sekce~\ref{subsubsec:pooling}).

Komunikaci ve skutečném čase využívá komponenta editoru pro synchronizaci textových operací, ale také pro zobrazení nových zpráv u dokumentu ve skutečném čase.
Komunikace probíhá po jednotlivých zprávách.
Každá má své jméno, může mít libovolný počet parametrů a funkci zpětného volání (tzv\. callback).